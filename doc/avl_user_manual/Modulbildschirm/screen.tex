\documentclass[a4paper,titlepage]{article}

\usepackage{color}
\usepackage{a4wide}
\usepackage{ngerman}
\parindent0cm
\newcommand{\bild}[1]{\bigskip \bigskip {\it an dieser stelle kommt ein bild} \dots (#1) \medskip}
\newcommand{\AVL}{\mbox{{\bf AVL-B"aume}}}

\usepackage{ifpdf}
\ifpdf 	\usepackage[pdftex]{graphicx}
\else	\usepackage[dvips]{graphicx}
\fi

\newcommand{\pfad}{../pics/}
%% F"ugt Bild an bestimmter Stelle, relativ zur Position des Befehls ein.
%% \userpic{Dateiname}{y-Offset}{x-Offset}{Bildvergr"o"serung}
\newcommand{\icon}[4]{ \vspace{#2 ex}	\hspace{#3 ex} 	\includegraphics[scale=#4]{\pfad #1}	\medskip}

%% F"ugt ein Bild mittig mit Bildunterschrift ein.
%% \centerpic{Dateiname}{Bildvergr"o"serung}{Untertitel}
\newcommand{\centerpic}[3]{
	\begin{center}
		\includegraphics[scale=#2]{\pfad #1} \\
		{\small #3}
	\end{center}
}

\begin{document}

\section{Die Arbeitsfl"ache}

Die Arbeit mit \AVL \ spielt sich auf der Arbeitsfl"ache ab. Sie bietet alle Funktionalit"aten 
des Moduls und ist in vier wichtige Bereiche unterteilt. 

\begin{itemize}
  \item {\sc Zeichenfl"ache} \\
		Der Baum und alle Algorithmen werden hier visualisiert.
  \item {\sc Infobereich} \\
		Wichtige Baumdaten wie Anzahl der Knoten, Baumh"ohe und Suchtiefe sind hier zu finden.
  \item {\sc Kontroll-Bereich} \\
		In diesem Bereich erfolgt der Start und die Steuerung der Algorithmen.
  \item {\sc Dokumentationsbereich} \\
		 Hier l"auft der Text zum jeweiligen Algorithmus mit. Der aktuelle Schritt wird dabei farbig hervorgehoben. 
		 Der Text ist dem Skript "`Algorithmen, Datenstrukuren und Programmierung"' von Prof.Vogler, 
		 Version vom 2. Oktober 2003, entnommen.
  \item {\sc Logbuch} \\
		 Hier werden erfolgte Einzelaktionen protokolliert und dabei der jeweils aktuelle Schritt farbig hervorgehoben.
\end{itemize}

Die Aufteilung zwischen dem unteren und dem oberen Bereich kann mit dem Schiebebalken ver"andert werden. 
Per Klick auf die schwarzen Pfeile k"onnen Sie den Textbereich wahlweise maximieren, um einen "Uberblick "uber 
den Algorithmustext zu gewinnen, oder minimieren, um die Zeichenfl"ache zu vergr"ossern.

\bigskip
\bigskip
\centerpic{prgmscreen}{0.7}{Die Arbeitsfl"ache des Moduls AVL-B"aume}

\end{document}