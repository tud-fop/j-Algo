\section{Die Arbeitsfl�che}
Die Arbeit mit \avl spielt sich auf der Arbeitsfl�che ab. Sie bietet alle Funktionalit�ten 
des Moduls und ist in f�nf wichtige Bereiche unterteilt:
\begin{itemize}
	\item \textsc{Zeichenfl�che}\\
		Der Baum und alle Algorithmen werden hier visualisiert.
	\item \textsc{Infobereich}\\
		Wichtige Baumdaten wie Anzahl der Knoten, Baumh�he und Suchtiefe sind hier zu finden.
	\item \textsc{Kontroll-Bereich}\\
		In diesem Bereich erfolgt der Start und die Steuerung der Algorithmen.
	\item \textsc{Dokumentationsbereich}\\
		Hier l�uft der Text zum jeweiligen Algorithmus mit. Der aktuelle Schritt wird dabei farbig hervorgehoben. Der Text ist dem Skript "`Algorithmen, Datenstrukturen und Programmierung"' von Prof. Vogler, Version vom 2. Oktober 2003, entnommen.
	\item \textsc{Logbuch}\\
		Hier werden erfolgte Einzelaktionen protokolliert und dabei der jeweils aktuelle Schritt farbig hervorgehoben.
\end{itemize}

Die Aufteilung zwischen dem unteren und dem oberen Bereich kann mit dem Schiebebalken ver�ndert werden. Per Klick auf die schwarzen Pfeile k�nnen Sie den Textbereich wahlweise maximieren, um einen �berblick �ber den Algorithmustext zu gewinnen, oder minimieren, um die Zeichenfl�che zu vergr��ern.
\newpage
\centerpic{avl/prgmscreen}{1}{Die Arbeitsfl�che des Moduls \avl}