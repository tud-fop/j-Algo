\section{Algorithmussteuerung}
Aufgabe des Moduls \avl ist es, Baumalgorithmen, wie das Einf�gen und L�schen von Knoten, zu visualisieren. Jeder Algorithmus ist in verschiedene Teilschritte unterteilt, die nacheinander angezeigt werden. Das Visualisieren erfolgt dabei durch das Zeichnen des Baumes, durch die Erkl�rung der Schritte im Dokumentationsbereich und im Logbuch und durch die Neuberechnung der baumspezifischen Daten, die im Infobereich pr�sentiert werden.\\
Nachdem Sie einen Algorithmus gestartet haben, verweilt er in einem Initialzustand und wartet auf Ihre Eingabe. Nun haben Sie die M�glichkeit, den Algorithmus in kleinen oder gro�en Schritten zu durchlaufen; Sie k�nnen ihn sofort beenden oder direkt abbrechen. Daf�r bietet die Algorithmussteuerung die entsprechenden Werkzeuge.

\subsection{Schritt-Pfeile}
Mittels der Schritt-Pfeile steuern Sie die Abfolge der Einzelschritte und bekommen so eine detaillierte Sicht auf die Arbeitsweise des Algorithmus. Das Programm bietet Ihnen die M�glichkeit, einen Teilschritt r�ckg�ngig zu machen und damit gewisse Abl�ufe zu wiederholen. Die Schritt-Pfeile, welche die R�ckg�ngigfunktion anbieten, weisen in ihrer Richtung nach links und sind dadurch intuitiv von den Vorw�rts-Pfeilen zu unterscheiden.\\
Zus�tzlich gibt es f�r jede Richtung einen gro�en und einen kleinen Schritt, der per Knopfdruck ausgef�hrt wird.

Kleine Schritte beim Einf�gen eines Knotens stellen Schl�sselvergleiche, Balancenberechnungen und Rotationen dar. Gro�e Schritte hingegen sind zum Beispiel das Suchen der Einf�gestelle, das Einf�gen an dieser und die gesamte Balancenaktualisierung.

\subsubsectiondoubleicon{Einzel-Schritt-Pfeile}{avl/icon_undo}{avl/icon_perform}
Ein Klick auf diese Buttons realisiert einen kleinen Algorithmusschritt zur�ck bzw. nach vorn.
\subsubsectiondoubleicon{Block-Schritt-Pfeile}{avl/icon_undo_blockstep}{avl/icon_perform_blockstep}
Ein Klick auf diese Buttons realisiert einen gro�en Schritt zur�ck bzw. nach vorn. Sollte der Algorithmusablauf an eine Stelle geraten, an der es nur noch einen kleinen Schritt nach vorn bzw. zur�ck gibt, so hat der Block-Schritt die selbe Funktionalit�t wie ein Einzel-Schritt.

\subsectiondoubleicon{Abbruch und Beenden-Buttons}{avl/icon_abort}{avl/icon_finish}
Klicken Sie auf den Beenden-Button \raisebox{-1ex}{\includegraphics[scale=0.8]{\pfad avl/icon_finish}} um den laufenden Algorithmus bis zum Ende auszuf�hren.\\
Klicken Sie auf den Abbruch-Button \raisebox{-1ex}{\includegraphics[scale=0.8]{\pfad avl/icon_abort}} um den laufenden Algorithmus abzubrechen. Der Baum hat danach den gleichen Status wie vor Beginn des Algorithmus.\\
Ist ein Algorithmus beendet, so steht Ihnen diese Option nicht mehr zur Verf�gung, weil nur der
\textit{laufende} Algorithmus abgebrochen werden kann.

\subsection{Animationsgeschwindigkeit}
Beim Generieren eines Zufallsbaumes haben Sie die Option, den Ablauf der Baumerzeugung als Animation ablaufen zu lassen. Starten Sie in diesem Modus, so beginnt die Animation sofort und kann mit dem Geschwindigkeitsregler schneller oder langsamer abgespielt werden. Zu Beginn steht dieser auf der mittleren Position. Verschieben Sie den Regler nach links, um die Animation zu verlangsamen bzw. nach rechts, um sie zu beschleunigen.\\
Eine Animation der anderen Algorithmenabl�ufe ist in dieser Version von \avl nicht integriert.

\bigskip
\centerpic{avl/animregler}{1}{Der Regler f�r die Animationsgeschwindigkeit}

\bigskip