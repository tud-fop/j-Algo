\documentclass[a4paper,titlepage]{article}

\usepackage{color}
\usepackage{a4wide}
\usepackage{enumerate}
\usepackage{amsmath}
\usepackage{ngerman}
\parindent0cm


\newtheorem{alg}{Algorithmus}

\usepackage{ifpdf}
\ifpdf 	\usepackage[pdftex]{graphicx}
\else	\usepackage[dvips]{graphicx}
\fi

\newcommand{\pfad}{../pics/}
%% F"ugt ein Bild mittig mit Bildunterschrift ein.
%% \centerpic{Dateiname}{Bildvergr"o"serung}{Untertitel}
\newcommand{\centerpic}[3]{
	\begin{center}
		\includegraphics[scale=#2]{\pfad #1} \\
		{\small #3}
	\end{center}
}

\begin{document}


\begin{appendix}


\section{Einleitung zu Datenstrukturen}
Eine h"aufige Anwendung auf gro"sen Datenmengen ist das Suchen:
Das Wiederfinden eines bestimmten Elements oder bestimmter
Informationen aus einer gro"sen Menge fr"uher abgelegter Daten. Um
die Suche zu vereinfachen, werden den (m"oglicherweise sehr
komplexen) gro"sen Datens"atzen eineindeutige Suchschl"ussel (Keys)
zugeordnet. Der Vergleich zweier solcher Schl"ussel ist in der
Regel viel schneller als der Vergleich zweier Datens"atze. Wegen
der Schl"usselzuordnung reicht es aus, alle vorkommenden
Algorithmen nur auf den Schl"usseln zu betrachten; in Wirklichkeit
verweisen erst die Schl"ussel auf die Datens"atze. \\ 
Eine grundlegende Idee ist nun, die Daten in Form einer Liste oder
eines Feldes abzulegen. Diese Datenstruktur ist h"ochst einfach.
Jedoch ist der Aufwand f"ur das Suchen relativ hoch, n"amlich
$O(n)$. Genauer gesagt ben"otigt man f"ur eine erfolglose Suche $n$
Vergleiche (bei $n$ Elementen in der Datenstruktur), 
denn man muss jedes Element "uberpr"ufen, und f"ur
eine erfolgreiche Suche durchschnittlich $(n+1)/2$ Vergleiche
durchf"uhren, da
nach jedem Element mit der gleichen Wahrscheinlichkeit gesucht
wird. \\
Eine Verbesserung dieses Verfahrens w"are, die Daten
sortiert abzulegen, was zu einer bin"aren Suche f"uhrt. Die
Suche hat jetzt nur noch Komplexit"at $O(\log_2{n})$, da bei jedem
Suchschritt das Feld halbiert wird. Der Nachteil ist, dass durch
die Sortierung das Einf"ugen erschwert wird, da unter Umst"anden
viele Datens"atze bewegt werden m"ussen. Das Verfahren sollte also
nur angewandt werden, wenn sehr wenige oder gar keine
Einf"ugeoperationen ausgef"uhrt werden m"ussen. Dann k"onnen die
Daten anfangs mit einem schnellen Verfahren sortiert werden und
m"ussen danach nicht mehr ver"andert werden. \\
Eine weitere Datenstruktur, die der Suchb"aume, wollen wir hier
vorstellen.





\section{\label{A}Suchb"aume}
Suchb"aume sind bin"are B"aume (jeder Knoten hat h"ochstens 2
Kinder) mit der Eigenschaft: \\ F"ur jeden Knoten gilt: alle
Schl"ussel im rechten Teilbaum sind gr"o"ser als der eigene
Schl"ussel und alle Schl"ussel im linken Teilbaum sind kleiner.
Diese Eigenschaft wird hier immer Suchbaumeigenschaft genannt. 

\centerpic{bsp_suchbaum}{4.5}{Ein Beispiel f"ur einen Suchbaum} 
An dieser Stelle noch eine Bemerkung zu den Schl"usseln: Die
Schl"ussel k"onnen Elemente einer beliebigen Menge sein, unter der
Bedingung, dass auf dieser Menge eine Ordnungsrelation definiert
ist. Die Ordnungsrelation wird offensichtlich f"ur die
Suchbaumeigenschaft ben"otigt, da dort die Begriffe "`kleiner"'
und "`gr"o"ser"' vorkommen. Eine h"aufig verwendete Menge sind die
nat"urlichen Zahlen mit ihrer normalen Ordnungsrelation $\leq$.
Alle Operationen auf Suchb"aumen kann man sich also anhand
der nat"urlichen Zahlen vorstellen.\\
Aus der Suchbaumeigenschaft kann man sich leicht rekursive
Algorithmen f"ur das Einf"ugen und Suchen in einem Suchbaum
herleiten: \\


\begin{alg} \label{search}
    (Suchen eines Schl"ussels $s$ in einem Suchbaum)
    \begin {enumerate}[(a)]
    \item Falls Teilbaum leer, dann Schl"ussel nicht im Baum
    vorhanden
    \item Falls $s$ gleich dem Schl"ussel des aktuellen Knoten,
    dann Suche erfolgreich.
    \item Falls $s$ gr"o"ser als Schl"ussel des aktuellen Knotens,
    dann suche (rekursiv) $s$ im rechten Teilbaum.
    \item Falls $s$ kleiner als Schl"ussel des aktuellen Knotens,
    dann suche (rekursiv) $s$ im linken Teilbaum.
    \end {enumerate}
\end{alg}


\begin{alg} \label{insert}
    (Einf"ugen eines Schl"ussels $s$ in einen Suchbaum)
    \begin {enumerate}[(a)]
    \item Falls Teilbaum leer, dann neuen Schl"ussel hier
    einf"ugen.
    \item Falls $s$ gleich dem Schl"ussel des aktuellen Knotens,
    dann Schl"ussel bereits vorhanden, Einf"ugen nicht n"otig.
    \item Falls $s$ gr"o"ser als Schl"ussel des aktuellen Knotens,
    dann f"uge (rekursiv) $s$ in den rechten Teilbaum ein.
    \item Falls $s$ kleiner als Schl"ussel des aktuellen Knotens,
    dann f"uge (rekursiv) $s$ in den linken Teilbaum ein.
    \end {enumerate}
\end{alg}


Mit Algorithmus \ref{insert} ergibt sich folgende Eigenschaft der
Struktur von Suchb"aumen: im Gegensatz zur sortierten Liste
h"angt die Struktur eines Baumes davon ab, in welcher Reihenfolge
die Elemente eingef"ugt werden. So erh"alt man verschieden B"aume,
wenn man $1, 2, 3, 4$ in dieser Reihenfolge und in der Reihenfolge
$3, 2, 4, 1$ einf"ugt. 


\centerpic{12345}{4}{Unterschiedliche Suchb"aume bei unterschiedlicher Einf"ugereihenfolge} 


Im ersten Fall erh"alt man einen Suchbaum, der zur linearen Liste
entartet ist. Das ist nicht nur in dieser speziellen Reihenfolge
so, es gibt viele M"oglichkeiten einen entarteten Baum zu
erzeugen. Der Algorithmus hat also zwei Nachteile: zum Einen kann
der Suchaufwand linear zur Anzahl der Knoten im Baum sein, was
keine Verbesserung zur linearen Liste darstellt; zum andern h"angt
die G"ute des Verfahrens von der Eingabefolge ab. Der Vorteil von
Suchb"aumen wird klar, wenn man einen "`vollen"' Baum betrachtet,
d.h. jeder Knoten hat entweder kein Kind (Blattknoten) oder zwei
Kinder (innerer Knoten). Dann hat sowohl das Suchen (unabh"angig
davon, ob der Schl"ussel im Baum vorhanden ist) als auch das
Einf"ugen eine Komplexit"at von $O(\log{n})$. 

\centerpic{vollerbaum}{5}{Ein Beispiel f"ur einen vollen Baum}
\medskip
Es gibt einige Algorithmen, wie man den Einf"ugealgorithmus
modifizieren kann, um die entarteten F"alle zu vermeiden und immer
einen nahezu vollen Baum hat. Einen davon, den Algorithmus nach
Adelson-Velskij und Landis (AVL), werden wir sp"ater betrachten. \\


Doch zun"achst wollen wir noch eine weitere wichtige Operation auf
Suchb"aumen untersuchen: das L"oschen. Leider ist es nicht ganz so
einfach wie Suche und Einf"ugen. \\ 
Zuerst muss der zu l"oschende
Schl"ussel gesucht werden. Ist der betreffende Knoten ein Blatt,
kann er einfach entfernt werden. Hat der zu l"oschende Knoten nur
ein Kind, kann er durch dieses ersetzt werden. Der schwierige Fall
ist, wenn er zwei Kinder hat. Damit die Suchbaumeigenschaft
erhalten bleibt, muss man ihn durch den n"achst gr"o"seren Schl"ussel
ersetzen. Der n"achst gr"o"sere Schl"ussel befindet sich
offensichtlich im rechten Teilbaum. \\


\centerpic{remove}{5}{L"oschen eines Knoten im Suchbaum}
\medskip
Nach dem Ersetzen bleibt die Suchbaumeigenschaft erhalten, weil
alle Schl"ussel aus dem linken Teilbaum ohnehin kleiner sind als
die aus dem rechten. Au"serdem sind auch alle Schl"ussel aus dem
rechten Teilbaum gr"o"ser, sonst w"are es nicht der
n"achst gr"o"sere Schl"ussel gewesen. Aus diesen "Uberlegungen
erh"alt man folgende verbale Beschreibung des
L"oschen-Algorithmus:

\newpage
\begin{alg} \label{delete}
(L"oschen eines Schl"ussels $s$ aus einem Suchbaum)
\begin{enumerate}[1.]
\item Suche s nach Algorithmus \ref{search}. Falls s nicht im Baum
enthalten, terminiert der Algorithmus.
\item
    \begin{enumerate}[(a)]
    \item Ist der zu l"oschende Knoten ein Blatt, dann entferne ihn
    einfach aus dem Baum.
    \item Hat der zu l"oschende Knoten nur ein Kind, dann ersetze
    ihn durch dieses.
    \item Sonst suche den kleinsten Schl"ussel im rechten
    Teilbaum: Gehe zum rechten Kind und dann immer zum
    linken Teilbaum, solange dieser nicht leer ist. Ersetze den zu
    l"oschenden Schl"ussel durch den des so gefundenen Knotens.
    Ersetze den gefundenen Knoten durch sein rechtes Kind.
    \end{enumerate}
\end{enumerate}
\end{alg}

Man kann anstelle des n"achst gr"o"seren Schl"ussels genausogut den
n"achstkleineren nehmen, der Algorithmus funktioniert trotzdem.
Zur Komplexit"at ist zu sagen, dass der Algorithmus maximal $h$
Vergleiche macht, wobei $h$ die H"ohe des Baumes ist. Auch hier ist
also die Komplexit"at abh"angig von der Struktur des Baumes.





\section{AVL-B"aume}
AVL-B"aume (benannt nach Adelson-Velskij und Landis) sind
spezielle Suchb"aume: In jedem Knoten unterscheiden sich die H"ohen
des linken Teilbaums und des rechten Teilbaums um h"ochstens 1. Um
diese Eigenschaft (AVL-Eigenschaft) abzusichern, wird f"ur jeden
Knoten ein Balancefaktor eingef"uhrt. Der Balancefaktor ist die
Differenz der H"ohen des rechten Teilbaums und des linken
Teilbaums. Also gilt f"ur jeden AVL-Baum, dass alle
Balancefaktoren aus $\{-1,0,1\}$ sind. B"aume mit AVL-Eigenschaft
sind niemals als lineare Liste entartet (, sofern sie denn mehr als
2 Knoten haben), sondern sind immer fast
vollst"andig. Zwischen der H"ohe $h$ eines Baumes und der Anzahl $n$
seiner Knoten besteht folgender Zusammenhang: $h\leq 2 \cdot
\log_2{n}$. Das bedeutet, dass f"ur das Suchen logarithmische
Komplexit"at garantiert werden kann (das Suchen erfolgt gem"a"s
Algorithmus \ref{search}). 


\centerpic{avlbsp}{5}{Ein Beispiel f"ur einen AVL-Baum mit Balancen}
\medskip
Jetzt muss noch untersucht werden, wie gro"s der zus"atzliche
Aufwand beim Einf"ugen ist, um die AVL-Eigenschaft zu wahren. In
jedem Fall wird der neue Knoten als Blatt eingef"ugt (nach
demselben Algorithmus wie bei Suchb"aumen). Dabei kann sich der
Balancefaktor "andern. Allerdings kann man sich leicht klarmachen,
dass das nur entlang des Suchpfades passieren kann. Die
Aktualisierung von Balancefaktoren erfolgt von der Einf"ugestelle
zur Wurzel. Hier der Algorithmus: 

\newpage
\begin{alg} \label{avlinsert}
(Algorithmus zum Einf"ugen eines Elements x in einen AVL-Baum)
\begin{enumerate}
\item F"uge das neue Element x als direkten Nachfolger des Knotens
n als Blatt ein, sodass die Suchbaumeigenschaft erf"ullt bleibt.
Aktualisiere n.balance.
\item Setze n auf den Vorg"angerknoten von
n.
    \begin{enumerate}[(a)]
    \item Falls x im linken Unterbaum von n eingef"ugt wurde
        \begin{enumerate}[(i)]
        \item wenn n.balance==1 dann n.balance=0 und gehe nach
        3.
        \item wenn n.balance==0, dann n.balance=-1 und gehe nach
        2.
        \item wenn n.balance==-1 und
            \begin{itemize}
            \item wenn n.left.balance==-1, dann Rechts(n)-Rotation.
            \item wenn n.left.balance==1 dann
            Links(n.left)-Rechts(n)-Rotation.
            \end{itemize}
            Gehe zu 3.
        \end{enumerate}
    \item Falls x im rechten Unterbaum von n eingef"ugt wurde
        \begin{enumerate}[(i)]
        \item wenn n.balance==-1 dann n.balance=0 und gehe nach
        3.
        \item wenn n.balance==0, dann n.balance=1 und gehe nach
        2.
        \item wenn n.balance==1 und
            \begin{itemize}
            \item wenn n.left.balance==1, dann Links(n)-Rotation.
            \item wenn n.left.balance==-1 dann
            Rechts(n.left)-Links(n)-Rotation.
            \end{itemize}
            Gehe zu 3.
        \end{enumerate}
    \end{enumerate}
    \item Gehe zur"uck zur Wurzel.
\end{enumerate}
\end{alg}

Zur Analyse dieses Algorithmus: Das reine Einf"ugen erfolgt
gem"a"s Einf"ugen im Suchbaum (Algorithmus \ref{insert}),
allerdings ist hier sichergestellt, dass sich die H"ohe
logarithmisch zur Anzahl der Knoten verh"alt, d.h. auch der
Aufwand f"ur das Einf"ugen ist garantiert logarithmisch. Wie
gesagt k"onnen sich jedoch Balancefaktoren ge"andert haben, sodass
die AVL-Eigenschaft nicht mehr erf"ullt ist. Das wird durch
sogenannte Rotationen behoben. Es gibt zwei Typen von Rotationen
-- Linksrotation und Rechtsrotation, jeweils um einen Knoten n. 

\centerpic{rotateleft}{5}{Linksrotation um den Knoten 65}  \medskip
\centerpic{rotateright}{5}{Rechtsrotation um den Knoten 58}  \medskip


Falls durch das Einf"ugen irgendwo ein Balancefaktor $2$($-2$)
entsteht (gr"o"sere "Anderungen k"onnen beim Einf"ugen eines
Knotens offensichtlich nicht auftreten), hei"st das, dass sich im
rechten (linken) Teilbaum die H"ohe um 1 erh"oht hat. Durch
Rotation(en) wie im Algorithmus angegeben, wird aber genau diese
H"ohe wieder reduziert. Somit ist klar, dass man maximal zweimal
rotieren muss, der Aufwand ist also noch ertr"aglich, im Gegensatz
zum L"oschen, wie man gleich sehen wird.

Genauso wie Einf"ugen ver"andert auch L"oschen eines Knotens aus
einem AVL-Baum die Balancefaktoren, also m"ussen auch hier
Rotationen ausgef"uhrt werden. Hier der Algorithmus:

\begin{alg} \label{avldelete}
(L"oschen eines Knotens aus einem AVL-Baum)
\begin{enumerate}
\item L"osche den Knoten analog zum L"oschen im Suchbaum
(Algorithmus \ref{delete}). Falls der Knoten ein Blatt war oder
nur einen linken Nachbarn hatte, setze aktuellen Knoten auf den
Vater. Sonst setze aktuellen Knoten auf den Vater des Knotens mit
dem n"achst gr"o"seren Schl"ussel. 

\item Berechne den Balancefaktor
des aktuellen Knotens neu. Falls
    \begin{enumerate}[(i)]
    \item Balance 2 und rechte Balance -1, dann
    Rechts(n.right)-Links(n)-Rotation.
    \item Balance 2 und rechte Balance nicht -1, dann
    Links(n)-Rotation.
    \item Balance -2 und linke Balance 1, dann
    Links(n.left)-Rechts(n)-Rotation.
    \item Balance -2 und linke Balance nicht 1, dann
    Rechts(n)-Rotation.
    \item sonst keine Rotation.
    \end{enumerate}
    Wiederhole diesen Schritt solange, bis die Wurzel erreicht
    ist.
\end{enumerate}
\end{alg}

Auch hier wollen wir den Aufwand des Algorithmus etwas genauer
untersuchen. Schritt 1 entspricht dem L"oschen aus dem Suchbaum,
nur garantiert mit logarithmischen Aufwand. Die Frage ist nun, ob,
wie beim Einf"ugen, der Algorithmus mit maximal zwei Rotationen
auskommt. Leider ist das nicht der Fall. Das liegt an der
erw"ahnten Eigenschaft der Rotationen, sie verringern die H"ohe
eines Teilbaums. Beim Einf"ugen war das gut, da durch das
Anh"angen eines Knotens gerade die H"ohe vergr"o"sert wurde. Hier
jedoch ist das unvorteilhaft, es kann passieren, dass man mehrere
Rotationen auf dem Weg zur Wurzel durchf"uhren muss. Die Anzahl
der Rotationen ist nur durch die H"ohe des Baums beschr"ankt. Das
ist in Anwendungsf"allen nicht w"unschenswert.

Bei zeitkritischen Anwendungen muss man also entweder auf einen
anderen Algorithmus ausweichen, oder Varianten wie etwa
Lazy-Delete implementieren, d.h. der Knoten wird nur als gel"oscht
markiert und wird sp"ater (wenn Zeit ist) aus dem Baum entfernt.



\begin{thebibliography}{99}
    \bibitem {sedgewick} R. Sedgewick, "`Algorithmen in C++"', Addison-Wesley, 5.
    Auflage, 1999
    \bibitem {vogler} Prof. Vogler, "`Vorlesungsskript Algorithmen,
    Datenstrukturen und Programmierung"', TU Dresden, 2003
    \bibitem {wiki} http://de.wikipedia.org/wiki/AVL-Baum
    \bibitem {uni-leipzig} http://dbs.uni-leipzig.de/de/skripte/ADS1/HTML/kap6-11.html
\end{thebibliography}





\end{appendix}

\end{document}

