


\section{Programmstart - Der Willkommensbildschirm}

\medskip
Nach Starten des Hauptprogramms JAlgo k"onnen Sie mit dem Men"upunkt {\sc "<Datei">\--"<Neu">\-"<AVL - B"aume">} ein 
neues Modul "offnen. Anschlie"send "offnet sich der Willkommensbildschirm des Moduls, der Ihnen verschiedene
M"oglichkeiten er"offnet. \\

\bigskip
\bigskip
\centerpic{welcomscreen}{0.5}{Der Willkommensbildschirm des Moduls AVL-B"aume}

\newpage
\subsection[Baum laden]{ }
\vspace{-4.3ex} {\bf {\large  \qquad \qquad Baum laden}} 

\icon{dir}{-4}{5}{0.33}

Mit Klick auf das Ordner-Symbol "offnet sich ein Dialogfenster, in dem Ihnen die M"oglichkeit 
gegeben wird, eine *.jalgo-Datei auszuw"ahlen, in welcher ein Baum gespeichert wurde.  \\

\centerpic{loaddialog}{0.45}{Das Dialogfenster zum Laden eines Baumes}

\medskip
\subsection[Baum von Hand erstellen]{} 
\vspace{-4.3ex} {\bf {\large  \qquad \qquad Baum von Hand erstellen}} 

\icon{hand}{-4}{5}{0.33}

Mit Klick auf das Hand-Symbol gelangen Sie sofort zur leeren Arbeitsfl"ache von \AVL .
Sie k"onnen jetzt mit der knotenweisen Generierung eines neuen Suchbaumes beginnen.

\medskip
\subsection[Zufallsbaum erstellen lassen]{}  
\vspace{-4.3ex} {\bf {\large  \qquad \qquad Zufallsbaum erstellen lassen}} 

\icon{wurfel}{-4}{5}{0.33}

Mit Klick auf das W"urfel-Symbol beginnen Sie die Generierung eines zuf"allig erzeugten Suchbaumes. In dem folgenden
Dialogfenster k"onnen Sie verschiedene Daten zum Baum und die Art der Visualisierung festlegen. 

\begin{itemize}
	\item {\bf Anzahl der Knoten} \\
	 Geben Sie hier die Anzahl der Knoten ein. Der entstehende Baum muss mindestens einen Knoten enthalten, h"ochstens aber 99. 
	\item {\bf AVL-Eigenschaft} \\
	 Aktivieren Sie dieses K"astchen, wenn der zu erstellende Baum die AVL-Eigenschaft besitzen soll.
	\item {\bf Visualisierung:} W"ahlen Sie hier die Art der Visualisierung der Erstellung aus.

	\begin{itemize}
		\item {\sc keine} \\ Der Baum wird sofort erstellt.
		\item {\sc schrittweise}\\ Jeder Algorithmusschritt kann von Ihnen per Hand best"atigt werden.
		\item {\sc automatisch}\\ Lassen Sie die Erstellung des Baumes als Animation ablaufen, die 
												Geschwindigkeit ist dabei einstellbar. 
	\end{itemize}
	Haben Sie schrittweise oder automatische Visualisierung gew"ahlt, k"onnen Sie den Ablauf jederzeit abbrechen. 
    Dabei wird das gerade aktive Knoteneinf"ugen abgebrochen, und der Baum steht mit entsprechend weniger Knoten zur Verf"ugung.
\end{itemize}
\bigskip

\centerpic{rgd}{0.7}{Eingabe der Zufallsbaumdaten}

\subsection[Willkommensbildschirm anzeigen]{}
\vspace{-4.3ex} {\bf {\large  \qquad \quad \ \ Willkommensbildschirm anzeigen}} 

\icon{icon-avl}{-4.2}{5.5}{1}

Mit Klick auf diesen Button in der Werkzeugleiste des Modulbildschirms kann der Willkommensbildschirm sp"ater jederzeit wieder
angezeigt werden. Dabei werden Sie evtl. gefragt, wie Sie mit Ihren "Anderungen verfahren wollen.  
Sollten sie Ihre "Anderungen nicht verwerfen, so wird eine neue Instanz des Moduls ge"offnet. \\

\centerpic{clearmessage}{0.6}{Dialog mit der Frage, ob der ganze Baum gel"oscht werden soll.}
