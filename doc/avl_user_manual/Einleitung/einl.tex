
%erster Abschnitt
\section{Einleitung}

Dieses Handbuch soll eine Einf"uhrung in die Arbeitsweise und die Funktionen des JAlgo-Moduls
\AVL \ darstellen. Bei JAlgo handelt es sich um ein Programm, das in erster Linie als Grundger"ust f"ur
Module gedacht ist, die sich mit der Visualisierung von Algorithmen besch"aftigen. \AVL \ im speziellen
realisiert die Darstellung von bin"aren Such- und AVL-B"aumen. F"ur eine Beschreibung dieser Baumtypen
lesen Sie bitte im Anhang \ref{A}. \\
Sowohl JAlgo als auch dieses Modul entstanden im Rahmen des externen Softwarepraktikums im Studiengang Informatik 
der TU Dresden f"ur den Lehrstuhl Programmierung. Das Einsatzgebiet soll vor allem die Vorlesung und das
studentische Lernen zu Hause umfassen. \\
\\
Bei JAlgo handelt es sich um eine freie Software, die beliebig oft kopiert werden darf.


\bigskip

%n"achster Abschnitt
\section{Technische Hinweise}
\subsection{Systemvoraussetzungen}

Folgende minimale Systemanforderungen werden f"ur den reibungslosen Einsatz von \AVL \ ben"otigt:

\begin{itemize}

\item IBM-kompatibler PC 

\item Mindestens 64 MB RAM

\item {\sc Windows} 98(SE)/ME/2000/XP , {\sc Linux} SuSE/Red Had 	

\item Java 2 Platform Standard Edition 5.0 	{\small (siehe: \textcolor{blue}{\underline{http://java.sun.com/}}) }

\item Maus und Tastatur	

\item Monitor mit einer Aufl"osung von mindestens 800x600 

\end{itemize}


\medskip
\subsection{Installation}

{\bf Windows} \\
Entpacken Sie nach dem Herunterladen das ZIP-komprimierte Archiv in einen Ordner Ihrer Wahl. 
In diesem Ordner finden Sie eine Datei namens "`j-algo.bat"'. 
"Offnen Sie diese Datei mit einem Doppelklick, und das Programm wird gestartet. \\

{\bf Unix}  \\
Entpacken Sie nach dem Herunterladen das ZIP-komprimierte Archiv in einen Ordner Ihrer Wahl. 
In diesem Ordner finden Sie eine Datei namens "`j-algo.sh"'. 
"Offnen Sie diese Datei mit einem Doppelklick, und das Programm wird gestartet.


\medskip
\subsection{Deinstallation}
Der komplette Programmordner kann jederzeit gefahrlos von der Festplatte gel"oscht werden.


\bigskip

%der dritte Abschnitt
\section{Funktions"ubersicht}

Hier eine kleine Zusammenstellung der Funktionen des Moduls:
\begin{itemize}
\item Visualisieren von bin"aren Suchb"aumen mit und ohne AVL-Eigenschaft
\item Einf"ugen, Suchen und L"oschen von Baumknoten 
\item Testen eines Baumes auf AVL-Eigenschaft
\item Generieren von zuf"alligen Suchb"aumen
\item Speichern und Laden von B"aumen
\item Informationen zum Baum und zu den laufenden Algorithmen

\end{itemize}

\bigskip
\bigskip
