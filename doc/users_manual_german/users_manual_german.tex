\documentclass{scrbook}
\usepackage[ngerman]{babel}
\usepackage[T1]{fontenc}
\usepackage[ansinew]{inputenc}
\usepackage{a4wide}

%verwenden von grafiken
\usepackage[dvipdf, final]{graphicx}

%verwenden von hyperlinks, stil derselben
\usepackage{color}
\definecolor{darkblue}{rgb}{0,0,0.5}

\usepackage{hyperref}
\hypersetup{
%	draft,										%hyperlinks ausschalten				
	colorlinks,									%hyperlinks farbig darstellen
	linkcolor   = darkblue,
	filecolor   = darkblue,
	urlcolor    = darkblue,
	citecolor   = darkblue,
	pdftitle    = {Handbuch},	%titel
	pdfsubject  = {j-Algo},			%thema
	pdfauthor   = {Alexander Claus, Matthias Schmidt},
	pdfkeywords = {Algorithmen, Visualisierung},
	pdfcreator  = {Distiller},
	pdfproducer = {LaTeX mit Hyperref-Package}}

%spezielle kommandos
%schreibweise des software-titels
\newcommand{\jalgo}{\mbox{\bfseries {\color{red}j}-Algo} }
%pfad zu den bildern
\newcommand{\pfad}{pics/}
%f�gt ein bild an einer bestimmten stelle, relativ zur position des befehls, ein.
%usage: \icon{dateiname}{y-offset}{x-offset}{bildvergr��erung}
\newcommand{\icon}[4]{
	\vspace{#2 ex}
	\hspace{#3 ex}
	\includegraphics[scale=#4]{\pfad #1}
}
%f�gt ein bild mittig mit bildunterschrift ein.
%usage: \centerpic{dateiname}{bildvergr��erung}{untertitel}
\newcommand{\centerpic}[3]{
	\begin{center}
		\includegraphics[scale=#2]{\pfad #1}\\
		{\small #3}
	\end{center}
}
%f�gt eine \subsection mit einem f�hrenden icon ein
%usage: \subsectionicon{text}{icon}
\newcommand{\subsectionicon}[2]{
	\subsection[#1]{\qquad #1}
	\icon{#2}{-5}{7}{1}
	\\
}
%f�gt eine \subsection mit 2 f�hrenden icons ein
%usage: \subsectiondoubleicon{text}{icon1}{icon2}
\newcommand{\subsectiondoubleicon}[3]{
	\subsection[#1]{\qquad \quad #1}
	\icon{#2}{-5}{7}{1}
	\icon{#3}{0}{-2}{1}
	\\
}
%f�gt eine \subsubsection* mit 2 f�hrenden icons ein
%usage: \subsubsectiondoubleicon{text}{icon1}{icon2}
\newcommand{\subsubsectiondoubleicon}[3]{
	\subsubsection*{\qquad \qquad #1}
	\icon{#2}{-4}{1}{1}
	\icon{#3}{0}{-2}{1}
	\\
}

\begin{document}

\begin{titlepage}
\centerpic{main/title}{1}{}
\vfill
\begin{flushright}
{\Huge \textbf{Benutzerhandbuch}}
\end{flushright}
\end{titlepage}

\newpage

\tableofcontents
\newpage

\part{Das Hauptprogramm}
\chapter{Das Modul Dijkstra}
\section{Einleitung}
Das Modul \dijkstra visualisiert den bekannten Algorithmus von E. W. Dijkstra zum Finden der k�rzesten Wege von einem Startknoten in einem Distanzgraphen. Der Algorithmus selbst ist unter anderem im Vorlesungsskript von Prof. Vogler "`Algorithmen, Datenstrukturen und Programmierung"' zu finden. Aber auch im Internet existieren zahlreiche Quellen dazu.

Soweit es m�glich gewesen ist, wurde beim Design des Moduls darauf geachtet, es weitgehend intuitiv und selbst-dokumentierend zu gestalten. Nichtsdestotrotz findet sich hier eine kurze
Einf�hrung in das \dijkstra - Modul.

\section{Funktions�bersicht}
Das Modul \dijkstra realisiert folgende Funktionen:
\begin{itemize}
	\item graphisches Erstellen / Bearbeiten eines Distanzgraphen
	\item Erstellen / Bearbeiten eines Graphen mittels Kanten- / Knotenliste oder Adjazenzmatrix
	\item Speichern und Laden von Graphen
	\item Visualisierung des Dijkstra-Algorithmus
\end{itemize}

\section{Modul starten}
Um das Modul zu starten, w�hlt man im Men� \textsc{<Datei>} das Submen� \textsc{<Neu>} und dann den Men�befehl \textsc{\dijkstra}. Im Hauptfenster erscheint nun die Oberfl�che des \dijkstra - Moduls im Eingabe-Modus.

\section{Symbolleiste}
Die Symbolleiste stellt die Funktionen \textsc{Speichern, Speichern unter, R�ckg�ngig} und \textsc{Wiederherstellen} bereit.
\newpage
\section{Grundfunktionen}
\jalgo bietet eine Reihe von Grundfunktionen, die unabh�ngig von den Modulen zur Verf�gung stehen, bzw. f�r jedes Modul die gleiche Bedeutung haben. Im Einzelnen sind das das �ffnen von Modulen sowie das Laden und Speichern von Sitzungsdaten. Die Grundfunktionen sind �ber die Werkzeugleiste oder den Men�punkt <\textsc{Datei}> erreichbar.

\subsectionicon{Neues Modul �ffnen}{main/icon_new}
Ein Klick auf den Button <\textsc{Neu}> in der Werkzeugleiste gibt Ihnen die M�glichkeit, ein beliebiges neues Modul zu �ffnen. Dabei wird ein Auswahldialog ge�ffnet, in welchem die installierten Module aufgelistet sind. Hier werden Ihnen au�erdem kurze Informationen zu diesen Modulen angezeigt. Sie k�nnen w�hlen, ob dieser Auswahldialog bei jedem Start des Programmes angezeigt werden soll oder nicht.\\
Alternativ dazu kann �ber das Men� \textsc{<Datei>$\rightarrow$<Neu>} das gew�nschte Modul geladen werden. Dies ist der schnellere Weg und zu empfehlen, wenn man bereits einen �berblick �ber die installierten Module hat.

\subsectionicon{Gespeicherte Sitzungsdaten laden}{main/icon_open}
Mit einem Klick auf den Button <\textsc{�ffnen}> erscheint ein Dialog zur Dateiauswahl. Hier haben Sie die M�glichkeit, eine Datei auszuw�hlen, in welcher modulspezifische Sitzungsdaten gespeichert wurden. Die Dateien, die von \jalgo gespeichert werden, tragen die Dateiendung \emph{"`.jalgo"'}.\\
Achtung: Da jedes Modul von \jalgo seine Daten in einer solchen Datei ablegt, kann man beim Blick auf die unge�ffnete Datei nicht erkennen, mit welchem Modul diese assoziiert wurde. Es wird jeweils das assoziierte Modul zu der geladenen Datei ge�ffnet. Achten Sie daher bei der Vergabe der Dateinamen auf m�glichst eindeutige Bezeichner.\\
Anmerkung: In einer sp�teren Version wird direkt bei der Dateiauswahl das zugeh�rige Modul mit angezeigt.
\centerpic{main/loaddialog}{0.45}{Das Dialogfenster zum �ffnen}

\medskip
\subsectiondoubleicon{Sitzungsdaten speichern}{main/icon_save}{main/icon_save_as}
Per Klick auf die Buttons <\textsc{Speichern}> und <\textsc{Speichern unter}> k�nnen Sie die Sitzungsdaten des gerade aktiven Moduls in einer Datei speichern. Wie beim Laden �ffnet sich auch hier ein Dialog zur Dateiauswahl, in welchem Sie Zielpfad und Name der neuen Datei eintragen k�nnen. Die Angabe der Dateiendung ist nicht n�tig, das Programm 
erg�nzt diese automatisch.\\
Je nach Implementierung des aktiven Moduls steht die Speicherfunktion nur zur Verf�gung, wenn gerade kein Algorithmus l�uft. Sollte noch ein Algorithmus aktiv sein, so beenden Sie diesen bitte vorher oder brechen ihn ab.

\subsection{Modul schlie�en}
Sie haben die M�glichkeit, jede Modulinstanz durch Klick auf das Kreuz der dazugeh�rigen Registerkarte zu schlie�en. Dabei werden Sie gegebenenfalls gefragt, ob Sie Ihre Arbeit speichern wollen. Um das gesamte Programm zu schlie�en, ist es nicht n�tig, die Module einzeln zu schliessen, das erledigt das Programm f�r Sie.

\bigskip
\begin{center}
	\raisebox{-7ex}{\includegraphics[scale=0.8]{\pfad main/closebutton}} \hfill
	\includegraphics[scale=0.55]{\pfad main/closemessage} \\
	{\small Der Knopf zum Schlie�en eines Moduls.} \hfill
	{\small Die Abfrage, ob die Daten gespeichert werden sollen.}
\end{center}

\subsectionicon{Einstellungen}{main/icon_prefs}
\jalgo bietet ein paar M�glichkeiten an, das Programm den Bed�rfnissen des Benutzers anzupassen. Den Dialog f�r die Grundeinstellungen erreichen Sie unter dem Men�punkt\\ \textsc{<Datei>$\rightarrow$<Einstellungen>}
\centerpic{main/prefsdialog}{0.8}{Der Dialog f�r die Grundeinstellungen}
Unter anderem kann hier die Sprache eingestellt werden. Au�erdem bietet sich die M�glichkeit, die Art der graphischen Oberfl�che zu �ndern, da unter manchen Betriebssystemen diverse Oberfl�chenelemente nicht immer vorteilhaft aussehen.

\subsectionicon{Hilfe}{main/icon_help}
Die Hilfe stellt ein wichtiges Nachschlagewerk f�r all diejenigen dar, die nicht auf Anhieb mit allen Funktionen von \jalgo und seinen Modulen klar kommen. Hier k�nnen Sie noch einmal eine genaue Beschreibung zu den einzelnen Programmelementen nachlesen.\\
Die Hilfe ist kontextspezifisch aufgebaut, d.h. ist ein Modul ge�ffnet, so wird in der Hilfe automatisch an die entsprechende Stelle gesprungen.\\
Sie erreichen die Hilfe �ber den Men�punkt \textsc{<Hilfe>$\rightarrow$<Inhalt>} oder indem Sie einfach auf die Taste <\textsc{F1}> Ihrer Tastatur dr�cken.

\subsection{Hinweis-Tipps}
Zus�tzlich wird zu den meisten Kontrollelementen, also Buttons, Men�eintr�ge, etc., ein kurzer Hinweistext neben dem Mauszeiger bzw. in der Statuszeile des Programmes angezeigt. Dies sollte als schnelle Hilfestellung den meisten Anforderungen gen�gen.

\newpage
\section{Impressum}
Die \jalgo Software wurde im Sommersemester 2004 von der Praktikumsgruppe SWT04-PROG1 im Rahmen des externen Softwarepraktikums entwickelt. Mitwirkende waren die
\paragraph{Teammitglieder}
\begin{itemize}
	\item Michael Pradel --- Chief of Algorithms
	\item Cornelius Hald --- Chief of Framework
	\item Malte Blumberg
	\item Stephan Creutz
	\item Christopher Friedrich
	\item Anne Kersten
	\item Hauke Menges
	\item Babett Schalitz
	\item Benjamin Scholz
	\item Marco Zimmerling
\end{itemize}
Die Webseite des Praktikums finden Sie unter \url{http://web.inf.tu-dresden.de/~swt04-p1/}.\\
Komplette �berarbeitung erfuhr die Software und die Dokumentation unter anderem durch Alexander Claus und Matthias Schmidt. Weitergehende Informationen �ber \jalgo erhalten Sie unter \url{http://j-algo.binaervarianz.de/}.

\part{Die Module}
\newcommand{\avl}{\mbox{\bfseries AVL-B�ume} }

\chapter{Das Modul Dijkstra}
\section{Einleitung}
Das Modul \dijkstra visualisiert den bekannten Algorithmus von E. W. Dijkstra zum Finden der k�rzesten Wege von einem Startknoten in einem Distanzgraphen. Der Algorithmus selbst ist unter anderem im Vorlesungsskript von Prof. Vogler "`Algorithmen, Datenstrukturen und Programmierung"' zu finden. Aber auch im Internet existieren zahlreiche Quellen dazu.

Soweit es m�glich gewesen ist, wurde beim Design des Moduls darauf geachtet, es weitgehend intuitiv und selbst-dokumentierend zu gestalten. Nichtsdestotrotz findet sich hier eine kurze
Einf�hrung in das \dijkstra - Modul.

\section{Funktions�bersicht}
Das Modul \dijkstra realisiert folgende Funktionen:
\begin{itemize}
	\item graphisches Erstellen / Bearbeiten eines Distanzgraphen
	\item Erstellen / Bearbeiten eines Graphen mittels Kanten- / Knotenliste oder Adjazenzmatrix
	\item Speichern und Laden von Graphen
	\item Visualisierung des Dijkstra-Algorithmus
\end{itemize}

\section{Modul starten}
Um das Modul zu starten, w�hlt man im Men� \textsc{<Datei>} das Submen� \textsc{<Neu>} und dann den Men�befehl \textsc{\dijkstra}. Im Hauptfenster erscheint nun die Oberfl�che des \dijkstra - Moduls im Eingabe-Modus.

\section{Symbolleiste}
Die Symbolleiste stellt die Funktionen \textsc{Speichern, Speichern unter, R�ckg�ngig} und \textsc{Wiederherstellen} bereit.
\newpage
\section{Programmstart - Der Willkommensbildschirm}
Nach Starten des Hauptprogramms \jalgo k�nnen Sie �ber den Button <\textsc{Neu}> oder mit dem Men�punkt \textsc{<Datei>$\rightarrow$<Neu>$\rightarrow$<\avl>} eine neue Instanz des Moduls \avl �ffnen. Anschlie�end �ffnet sich der Willkommensbildschirm des Moduls, der Ihnen verschiedene M�glichkeiten er�ffnet.\\
\bigskip
\centerpic{avl/welcomscreen}{0.5}{Der Willkommensbildschirm des Moduls \avl}
\bigskip

\subsectionicon{Baum laden}{avl/welcome_load}
Mit Klick auf das Ordner-Symbol �ffnet sich ein Dialogfenster, in dem Ihnen die M�glichkeit 
gegeben wird, eine \emph{"`*.jalgo"'} - Datei auszuw�hlen, in welcher ein Baum gespeichert wurde.\\
Im Prinzip ist die Bedeutung dieses Buttons die gleiche wie des <\textsc{�ffnen}>-Buttons in der Werkzeugleiste. Der Unterschied besteht darin, dass der Button in der Werkzeugleiste eine neue Modulinstanz �ffnet, in welcher die Datei geladen wird, der Button im Startbildschirm von \avl jedoch die Datei in die aktuell ge�ffnete Modulinstanz l�dt.

\subsectionicon{Baum von Hand erstellen}{avl/welcome_manual}
Mit Klick auf das Hand-Symbol gelangen Sie sofort zur leeren Arbeitsfl�che des Moduls \avl.
Sie k�nnen jetzt mit der knotenweisen Generierung eines neuen Suchbaumes beginnen.

\subsectionicon{Zufallsbaum erstellen lassen}{avl/welcome_random}
Mit Klick auf das W�rfel-Symbol beginnen Sie die Generierung eines zuf�llig erzeugten Suchbaumes. In dem folgenden Dialogfenster k�nnen Sie verschiedene Daten zum Baum und die Art der Visualisierung festlegen. 
\centerpic{avl/rgd}{1}{Eingabe der Zufallsbaumdaten}
\begin{itemize}
	\item {\bf Anzahl der Knoten}\\
	 Geben Sie hier die Anzahl der Knoten ein. Der entstehende Baum muss mindestens einen Knoten enthalten, h�chstens aber 99. 
	\item {\bf AVL-Eigenschaft}\\
	 Aktivieren Sie dieses K�stchen, wenn der zu erstellende Baum die AVL-Eigenschaft besitzen soll.
	\item {\bf Visualisierung}\\
	W�hlen Sie hier die Art der Visualisierung der Erstellung aus.
	\begin{itemize}
		\item {\sc keine}\\ Der Baum wird sofort erstellt.
		\item {\sc schrittweise}\\ Jeder Algorithmusschritt kann von Ihnen per Hand best�tigt werden.
		\item {\sc automatisch}\\ Lassen Sie die Erstellung des Baumes als Animation ablaufen, die Geschwindigkeit ist dabei einstellbar. 
	\end{itemize}
	Haben Sie schrittweise oder automatische Visualisierung gew�hlt, k�nnen Sie den Ablauf jederzeit abbrechen. 
    Dabei wird das gerade aktive Knoteneinf�gen abgebrochen, und der Baum steht mit entsprechend weniger Knoten zur Verf�gung.
\end{itemize}

\subsectionicon{Willkommensbildschirm anzeigen}{avl/logo}
Mit Klick auf diesen Button in der Werkzeugleiste des Modulbildschirms kann der Willkommensbildschirm sp�ter jederzeit wieder angezeigt werden. Dabei werden Sie eventuell gefragt, wie Sie mit Ihren �nderungen verfahren wollen. Sollten Sie Ihre �nderungen nicht verwerfen wollen, so wird eine neue Instanz des Moduls ge�ffnet.
\centerpic{avl/clearmessage}{1}{Dialog mit der Frage, ob der ganze Baum gel�scht werden soll.}

\bigskip
\section{Der Arbeitsbereich}
\label{sec:DerArbeitsbereich}

Der Arbeitsbereich ist untergliedert in drei Bereiche, die Ihnen Zugriff auf alle wesentlichen Funktionen in den Algorithmus-Phasen erm�glicht:

\bigskip
\centerpic{kmp/phase1screen}{0.5}{Phase 1 Bildschirm}
\bigskip

   
\begin{enumerate}
	\item  \textbf{Steuerung}\newline
      Hier wird der Algorithmus gestartet und gesteuert.
   \item \textbf{Visualisierung}\newline
      Hier werden die Verschiebetabelle bzw. der Text dargestellt.
   \item \textbf{Dokumentation}\newline
      Hier gibt es verschiedene Perspektiven, die miteinander kombiniert werden k�nnen: 			Quellcode, Text, Protokoll.
\end{enumerate}

\section{Anzeigeoptionen}
\label{sec:Anzeigeoptionen}

\subsection{Skalierung}
\label{sec:Skalierung}

Die Gr��e der Elemente des Visualisierungs- und Dokumentationsbereichs kann eingestellt werden, klicken Sie daf�r auf den Schieberegler im Visualisierungsbereich und ziehen Sie ihn nach oben bzw. unten.

\centerpic{kmp/slide}{0.5}{Auswirkung des Schiebereglers zur Skalierung}
\bigskip
\subsection{Aufteilung der Bereiche}
\label{sec:Aufteilung der Bereiche}

Die Aufteilung zwischen dem Visualisierungs- und dem Dokumentationsbereich kann mit einem Schiebebalken ver�ndert werden. Per Klick auf die Kante k�nnen Sie die Grenze verschieben.

\centerpic{kmp/move}{0.5}{Auswirkung des Schiebereglers}
\bigskip

\subsection{Beamer - Modus}
\label{sec:BeamerModus}

Der Beamer-Modus erm�glicht das schnelle Einstellen von Anzeigeoptionen, die die Pr�sentation in Vorlesungen oder �hnlichen Veranstaltungen beg�nstigen. Dieser Modus ist zu finden unter 'Knuth Morris Pratt' => 'Beamermodus'.
Die Anzeige ist f�r die Aufl�sung 1024x768 optimiert und vergr��ert die Elemente um den Faktor 1,5. Im Dokumentationsbereich wird die Perspektive 'Code' angezeigt.
Ist der Modus aktiv, so erscheint vor diesem Men�eintrag ein H�kchen. Um den Modus wieder auszuschalten, entfernen Sie einfach den Haken per Klick.

\centerpic{kmp/beamermod}{0.5}{Der Beamermodus}
\bigskip
\subsection{Zyklus - Anzeige}
\label{sec:ZyklusAnzeige}

Im Visualisierungsbereich der Phase 1 'Generierung der Verschiebetabelle' k�nnen optional die Zyklen angezeigt werden, setzen Sie dazu das H�kchen 'Zyklen' im Steuerungsbereich. Es werden maximal drei Zyklen gleichzeitig angezeigt.

\centerpic{kmp/cycle}{0.5}{Beispiel f�r Zyklen}
\bigskip
\section{Legende}
\label{sec:Legende}

\subsection{Elemente in der Phase 'Generierung der Verschiebetabelle'}
\label{sec:ElementeInDerPhaseGenerierungDerVerschiebetabelle}

\bigskip
\centerpic{kmp/phase1screen_ex1}{0.5}{Beispiel f�r Phase 1}
\bigskip
\textit{Visualisierungsbereich}\newline
\textbf{Pfeil schwarz mit wei�em 'P' }- Zeiger auf die Variable 'patpos', die Patternposition\newline
\textbf{Pfeil wei� mit schwarzem 'V' }- Zeiger auf die Variable 'VglInd', der Vergleichsindex\newline
\textbf{schwarzer Pfeil �ber der Tabelle }- die verglichenen Zeichen\newline
\textbf{gelber Hintergrund von Zellen} - Zellenkopf\newline
\textbf{schwarzer Rahmen um Zelle in der Zeile 'Index' }- aktuell kopierte Verschiebeinformation\newline
\textbf{roter Rahmen um Zellen }- negative boolesche Bedingung\newline
\textbf{lila Strich }- Zyklen, die im Pattern auftreten
\newline\newline
\textit{Schriftfarben}\newline
\textbf{blau} - aktuell geschriebene Verschiebeinformation (entspricht Zuweisungsstatement im Code)\newline
\textbf{rot} - die verglichenen Zeichen stimmen nicht �berein (entspricht booleschem Statement im Code)\newline
\textbf{gr�n} - die verglichenen Zeichen stimmen �berein (entspricht booleschem Statement im Code)

\subsection{Elemente in der Phase 'Suchen im Suchtext'}
\label{sec:ElementeInDerPhaseSuchenImSuchtext}
 \bigskip
\centerpic{kmp/phase2screen_ex1}{0.5}{Beispiel f�r diese Phase}
 \bigskip
\textit{Visualisierungsbereich}\newline
\textbf{lila Rahmen }- das Sichtfenster, welches die aktuell betrachteten Zeichen des Textes und des Patterns justiert
\newline\newline
\textit{Schriftfarben}\newline
\textbf{blau} - aktuell geschriebene Verschiebeinformation (entspricht Zuweisungsstatement im Code)

\subsection{Im Dokumentationsbereich}
\label{sec:ImDokumentationsbereich}

\bigskip
\centerpic{kmp/phase2screen_ex2}{0.5}{Beispiel f�r diese Phase}
\bigskip
\textit{Perspektive 'Code'}\newline
\textbf{roter Hintergrund }- negative boolesche Bedingung\newline
\textbf{gr�ner Hintergrund} - positive boolesche Bedingung\newline
\textbf{blauer Hintergrund }- sonstige Statements
\newline\newline
\textit{Perspektive 'Protokoll'}\newline
\textbf{blaue Schrift} - zuletzt vollzogener Schritt\newline
\textbf{schwarze Schrift} - f�r den aktuellen Stand irrelevante vorausgegangene Schritte
\newline\newline
\textit{Perspektive 'Suchtext'}\newline
\textbf{gelber Hintergrund} - der Textausschnitt, an dem das Pattern momentan anliegt

\section{Modulfunktionen}
\label{sec:Modulfunktionen}

\subsection{Eingabe von Pattern und Text}
\label{sec:EingabeVonPatternUndText}

\subsubsection{Pattern eingeben}
\label{sec:PatternEingeben}


\centerpic{kmp/pattern1}{0.5}{Pattern manuell eingeben}
\bigskip
Das Pattern kann manuell eingegeben werden, hierbei ist die maximale L�nge von 10 Zeichen zu beachten. Das Alphabet kann aus beliebigen Zeichen bestehen. Die Eingabe erfolgt im Steuerungsbereich in die Eingabezeile 'Pattern'. Um das Pattern zu �bernehmen klicken Sie auf 'Setzen'.

\subsubsection{Suchtext eingeben}


Der Text unterliegt keinen Beschr�nkungen und kann manuell eingegeben oder aus einer vorhandenen *.txt-Datei importiert werden.
Die Eingabe erfolgt im Steuerungsbereich, klicken Sie hierf�r auf 'Eingeben'.

\centerpic{kmp/text1}{0.5}{Text manuell eingeben}
\bigskip
Es �ffnet sich ein Fenster, in dem Sie den Text manuell eingeben (oder auch per Copy und Paste einf�gen) oder eine *.txt-Datei laden k�nnen. Um den Text zu �bernehmen klicken Sie auf 'Anwenden'.

\centerpic{kmp/text1}{0.5}{Text laden oder eingeben}
\bigskip
\subsection{Generieren von Pattern und passenden Texten}
\label{sec:GenerierenVonPatternUndPassendenTexten}

\subsubsection{Pattern eingeben}

\centerpic{kmp/pattern2}{0.5}{Pattern generieren lassen}
\bigskip
Neben der manuellen Eingabe des Patterns gibt es die M�glichkeit, ein Pattern generieren zu lassen. Klicken Sie daf�r auf den Knopf 'Zufall' (im Steuerungsbereich neben der Eingabezeile 'Pattern') und w�hlen Sie im folgenden Fenster die Kardinalit�t des zu nutzenden Alphabets und die gew�nschte L�nge des Patterns aus. Das Pattern wird mit Klick auf 'Anwenden' geniert und Sie kehren zum Arbeitsbereich zur�ck.

\centerpic{kmp/pattern3}{0.5}{Pattern generieren lassen}
\bigskip
\subsubsection{Suchtext eingeben}

\centerpic{kmp/text3}{0.5}{Suchtext generieren lassen}
\bigskip
Um einen zum Pattern passenden Text generieren zu lassen, klicken Sie auf 'Generieren' (im Steuerungsbereich neben 'Text'). Dadurch wird eine Text erstellt, der das Pattern enth�lt und nur aus den im Pattern genutzten Zeichen besteht. Dieser Text wird sofort �bernommen, Sie haben aber die M�glichkeit, ihn zu bearbeiten, indem Sie auf 'Eingeben' klicken.


\subsection{Generierung der Verschiebetabelle}
\label{sec:GenerierungDerVerschiebetabelle}


Wenn ein Pattern gesetzt wurde, kann die Verschiebetabelle erstellt werden. Um den Algorithmus zu starten und zu steuern, benutzen Sie die Pfeile im Steuerungsbereich.

\centerpic{kmp/control}{0.5}{Algorithmussteuerung}
\bigskip
Das Pattern kann nun noch erweitert werden ohne den aktuellen Algorithmus zu unterbrechen, geben Sie hierf�r das gew�nschte Zeichen in das Eingabefeld "Erweiterung" im Steuerungsbereich ein. Sollte das Pattern bereits die maximale L�nge haben, ist eine Erweiterung nicht m�glich.

\centerpic{kmp/expand}{0.5}{Pattern erweitern}
\bigskip
Nachdem die Verschiebetabelle erstellt wurde, k�nnen Pattern und Verschiebetabelle f�r eine Suche im Text genutzt werden. Klicken Sie hierf�r auf das Feld 'Weiter zu Phase 2', welches erst am Ende des Algorithmus im Steuerungsbereich erscheint.

\centerpic{kmp/go_p2}{0.5}{Von Phase 1 in Phase 2 wechseln}
\bigskip
\subsection{Suchen im Suchtext}
Erst wenn Pattern und Text gesetzt sind, kann das Anwenden der Verschiebetabelle an einem Text durchgef�hrt werden. Um Pattern und Text zu setzen, haben Sie verschiedene M�glichkeiten, siehe: Eingabe von Pattern und Text , Generieren von Pattern und passenden Texten.
Um den Algorithmus zu starten und zu steuern, benutzen Sie die Pfeile im Steuerungsbereich.

\centerpic{kmp/control}{0.5}{Algorithmussteuerung}
\bigskip
Sollten Sie w�hrend der Durchf�hrung des Algorithmus das Pattern und/oder den Text �ndern, wird die aktuelle Durchf�hrung des Algorithmus unterbrochen.
\bigskip
Der Suchvorgang wird beendet, wenn das Pattern gefunden wurde oder das Textende erreicht wurde, es erscheint eine Meldung dazu im Steuerungsbereich.

\centerpic{kmp/endnegativ}{0.5}{Nachricht, dass Pattern nicht gefunden wurde}
\bigskip
\centerpic{kmp/endpositiv}{0.5}{Nachricht, dass Pattern gefunden wurde}
\bigskip
\subsection{�ffnen einer KMP-Sitzung}
\label{sec:�ffnenEinerKMPSitzung}


Mit Klick auf das Ordner-Symbol �ffnet sich ein Dialogfenster, in dem Ihnen die M�glichkeit gegeben wird, eine '*.jalgo' - Datei auszuw�hlen, in welcher eine KMP-Sitzung gespeichert wurde.

\centerpic{kmp/load}{0.5}{Dialogfenster zum Laden einer Sitzung}
\bigskip

\subsection{Pr�sentation von Lernbeispielen}
Es stehen Ihnen zehn Beispiele zur Auswahl, die jeweils eine besondere Eigenschaft des KMP-Algorithmus repr�sentieren. Wenn Sie mit dem Mauszeiger �ber die Beispiele fahren, werden diese Eigenschaften kurz beschrieben. W�hlen Sie das gew�nschte Lernbeispiel aus und klicken Sie auf 'Laden' um das Beispiel zu starten.\newline
Die Beispiele bestehen aus Pattern- und passendem Text.

\centerpic{kmp/learning}{0.5}{Pr�sentation von Lernbeispielen}
\bigskip
Die Steuerung der Lernbeispiele erfolgt durch die Pfeile im Steuerungsbereich.


\section{Algorithmussteuerung}
Aufgabe des Moduls \avl ist es, Baumalgorithmen, wie das Einf�gen und L�schen von Knoten, zu visualisieren. Jeder Algorithmus ist in verschiedene Teilschritte unterteilt, die nacheinander angezeigt werden. Das Visualisieren erfolgt dabei durch das Zeichnen des Baumes, durch die Erkl�rung der Schritte im Dokumentationsbereich und im Logbuch und durch die Neuberechnung der baumspezifischen Daten, die im Infobereich pr�sentiert werden.\\
Nachdem Sie einen Algorithmus gestartet haben, verweilt er in einem Initialzustand und wartet auf Ihre Eingabe. Nun haben Sie die M�glichkeit, den Algorithmus in kleinen oder gro�en Schritten zu durchlaufen; Sie k�nnen ihn sofort beenden oder direkt abbrechen. Daf�r bietet die Algorithmussteuerung die entsprechenden Werkzeuge.

\subsection{Schritt-Pfeile}
Mittels der Schritt-Pfeile steuern Sie die Abfolge der Einzelschritte und bekommen so eine detaillierte Sicht auf die Arbeitsweise des Algorithmus. Das Programm bietet Ihnen die M�glichkeit, einen Teilschritt r�ckg�ngig zu machen und damit gewisse Abl�ufe zu wiederholen. Die Schritt-Pfeile, welche die R�ckg�ngigfunktion anbieten, weisen in ihrer Richtung nach links und sind dadurch intuitiv von den Vorw�rts-Pfeilen zu unterscheiden.\\
Zus�tzlich gibt es f�r jede Richtung einen gro�en und einen kleinen Schritt, der per Knopfdruck ausgef�hrt wird.

Kleine Schritte beim Einf�gen eines Knotens stellen Schl�sselvergleiche, Balancenberechnungen und Rotationen dar. Gro�e Schritte hingegen sind zum Beispiel das Suchen der Einf�gestelle, das Einf�gen an dieser und die gesamte Balancenaktualisierung.

\subsubsectiondoubleicon{Einzel-Schritt-Pfeile}{avl/icon_undo}{avl/icon_perform}
Ein Klick auf diese Buttons realisiert einen kleinen Algorithmusschritt zur�ck bzw. nach vorn.
\subsubsectiondoubleicon{Block-Schritt-Pfeile}{avl/icon_undo_blockstep}{avl/icon_perform_blockstep}
Ein Klick auf diese Buttons realisiert einen gro�en Schritt zur�ck bzw. nach vorn. Sollte der Algorithmusablauf an eine Stelle geraten, an der es nur noch einen kleinen Schritt nach vorn bzw. zur�ck gibt, so hat der Block-Schritt die selbe Funktionalit�t wie ein Einzel-Schritt.

\subsectiondoubleicon{Abbruch und Beenden-Buttons}{avl/icon_abort}{avl/icon_finish}
Klicken Sie auf den Beenden-Button \raisebox{-1ex}{\includegraphics[scale=0.8]{\pfad avl/icon_finish}} um den laufenden Algorithmus bis zum Ende auszuf�hren.\\
Klicken Sie auf den Abbruch-Button \raisebox{-1ex}{\includegraphics[scale=0.8]{\pfad avl/icon_abort}} um den laufenden Algorithmus abzubrechen. Der Baum hat danach den gleichen Status wie vor Beginn des Algorithmus.\\
Ist ein Algorithmus beendet, so steht Ihnen diese Option nicht mehr zur Verf�gung, weil nur der
\textit{laufende} Algorithmus abgebrochen werden kann.

\subsection{Animationsgeschwindigkeit}
Beim Generieren eines Zufallsbaumes haben Sie die Option, den Ablauf der Baumerzeugung als Animation ablaufen zu lassen. Starten Sie in diesem Modus, so beginnt die Animation sofort und kann mit dem Geschwindigkeitsregler schneller oder langsamer abgespielt werden. Zu Beginn steht dieser auf der mittleren Position. Verschieben Sie den Regler nach links, um die Animation zu verlangsamen bzw. nach rechts, um sie zu beschleunigen.\\
Eine Animation der anderen Algorithmenabl�ufe ist in dieser Version von \avl nicht integriert.

\bigskip
\centerpic{avl/animregler}{1}{Der Regler f�r die Animationsgeschwindigkeit}

\bigskip
\section{Dokumentation}
Da das Modul \avl vor allem zu Lehr- und Lernzwecken eingesetzt werden soll, ist eine detaillierte Dokumentation der Algorithmen unumg�nglich. F�r die Einzelheiten des Algorithmustextes steht ein Auszug aus dem Vorlesungsskript von Prof. Vogler zur Verf�gung. Ein Logbuch in der rechten unteren Ecke des Bildschirmes f�hrt Protokoll �ber den Stand und die Beschaffenheit des einzelnen Algorithmusteilschrittes.\\
Zu guter Letzt wird ein Infobereich angeboten, in dem wichtige Baumdaten zusammengefasst sind.

\subsection{Skript}
Der Dokumentationsbereich, der das Skript enth�lt, befindet sich am unteren Bildschirmrand. Es handelt sich hierbei um einen Auszug des Skripts zur Vorlesung "`Algorithmen und Datenstrukturen"' von Prof. Vogler (TU Dresden), Version vom 2. Oktober 2003. Im Rahmen dieser Vorlesung soll das Modul vorwiegend eingesetzt werden.\\
Bei dem jeweils aktuellen Algorithmustext handelt es sich um die Aktion, die als n�chstes im Ablauf des Algorithmus erfolgen wird. Sie wird rot markiert angezeigt.

\subsection{Logbuch}
Das Logbuch ist eine weitere M�glichkeit, den Ablauf des Algorithmus zu verfolgen. Es bezieht sich in erster Linie auf baumspezifische Daten und verwendet zum Beispiel konkrete Schl�sselwerte, anhand deren die Aktionen des Algorithmus besser verstanden werden sollen.\\
Auch hier wird der aktuelle Eintrag rot markiert dargestellt. Dieser bezieht sich aber auf die zuletzt ausgef�hrte Aktion.

\subsection{Infobereich}
Der Infobereich ist haupts�chlich daf�r gedacht, Ihnen schnell wichtige Daten des Baumes bereit zu stellen. Hier finden Sie folgende Punkte:
\begin{itemize}
	\item \textsc{Anzahl der Knoten}\\
		Dieser Punkt fasst f�r Sie die Anzahl der Knoten im Baum zusammen.
	\item \textsc{Baumh�he}\\
		Hier finden Sie die Anzahl der Level des Baumes.
	\item \textsc{Durchschnittliche Suchtiefe}\\
		Dieser Wert berechnet sich durch die Summe der Level aller Knoten geteilt durch die Anzahl dieser. Der Wert ist ein Indiz daf�r, wie gut der Baum ausbalanciert ist bzw. wie gro� der Suchaufwand im Durchschnitt ist.
\end{itemize}

\bigskip
\section{Zusatzfunktionen}
Dieses Kapitel widmet sich den Eastereggs des Moduls \avl.\\
Sollten Sie sich lieber selber gerne auf die Suche nach diesen Zusatzfunktionen machen wollen, so �berspringen Sie besser dieses Kapitel.\\
F�r alle Anderen folgt nun eine �bersicht zum Baumnavigator und dem Beamermodus.

\subsection{Navigator}
Der Navigator ist eine kleine, versteckte Zusatzfunktion, die die Arbeit mit gro�en B�umen erheblich vereinfachen kann. Er stellt eine willkommene Hilfe f�r das Scrollen der Zeichenfl�che dar, ist aber nicht so einfach zu finden.
\begin{itemize}
	\item Wenn der Baum, der auf der Zeichenfl�che angezeigt wird, zu gro� f�r diese wird, so erscheinen Schiebebalken, mit denen Sie den Bildausschnitt verschieben k�nnen.
	\item Klicken Sie nun auf das kleine Quadrat in der rechten unteren Ecke der Zeichenfl�che, genau zwischen den beiden Schiebebalken. Halten Sie dabei die linke Maustaste gedr�ckt.
	\item Eine kleine �bersichtskarte des Baumes mit einem Ausschnittfenster erscheint. Bewegen Sie die Maus (mit gedr�ckter Taste) und das Ausschnittfenster, das den Bildschirminhalt der Zeichenfl�che repr�sentiert, folgt Ihren Bewegungen.
\end{itemize}
\bigskip
\begin{center}
	\includegraphics[scale=0.7]{\pfad avl/navigator1} \hfill
	\includegraphics[scale=0.7]{\pfad avl/navigator2} \\
	Ein Klick auf das kleine K�stchen... \hfill ...�ffnet den Navigator!
\end{center}

\subsectionicon{Beamermodus}{avl/icon_beamer}
Der Beamermodus ist in erster Linie f�r die Pr�sentation in Vorlesungen oder �hnlichen Veranstaltungen gedacht. Ist dieser Modus aktiv, so werden die Knoten des Baumes und die Eintr�ge des Logbuches vergr��ert dargestellt. Der Algorithmustext aus dem Skript von Prof. Vogler bleibt dabei unver�ndert, weil davon ausgegangen wird, dass die interessierten Studenten der Vorlesung �ber ein (eventuell aktuelleres) Skript verf�gen.\\
Sie erreichen den Beamermodus �ber den Men�punkt \textsc{<\avl>} $\rightarrow$ \textsc{<Beamermodus>}. Ist der Modus aktiv, so erscheint neben diesem Men�eintrag ein H�kchen. Um den Modus wieder auszuschalten, entfernen Sie einfach den Haken per Klick.
\bigskip
\centerpic{avl/beamermenu}{0.5}{Das Men� \textsc{<\avl>} mit dem Eintrag \textsc{<Beamermodus>}}

\newpage
\section{Impressum}
Die \jalgo Software wurde im Sommersemester 2004 von der Praktikumsgruppe SWT04-PROG1 im Rahmen des externen Softwarepraktikums entwickelt. Mitwirkende waren die
\paragraph{Teammitglieder}
\begin{itemize}
	\item Michael Pradel --- Chief of Algorithms
	\item Cornelius Hald --- Chief of Framework
	\item Malte Blumberg
	\item Stephan Creutz
	\item Christopher Friedrich
	\item Anne Kersten
	\item Hauke Menges
	\item Babett Schalitz
	\item Benjamin Scholz
	\item Marco Zimmerling
\end{itemize}
Die Webseite des Praktikums finden Sie unter \url{http://web.inf.tu-dresden.de/~swt04-p1/}.\\
Komplette �berarbeitung erfuhr die Software und die Dokumentation unter anderem durch Alexander Claus und Matthias Schmidt. Weitergehende Informationen �ber \jalgo erhalten Sie unter \url{http://j-algo.binaervarianz.de/}.
\newcommand{\avl}{\mbox{\bfseries AVL-B�ume} }

\chapter{Das Modul Dijkstra}
\section{Einleitung}
Das Modul \dijkstra visualisiert den bekannten Algorithmus von E. W. Dijkstra zum Finden der k�rzesten Wege von einem Startknoten in einem Distanzgraphen. Der Algorithmus selbst ist unter anderem im Vorlesungsskript von Prof. Vogler "`Algorithmen, Datenstrukturen und Programmierung"' zu finden. Aber auch im Internet existieren zahlreiche Quellen dazu.

Soweit es m�glich gewesen ist, wurde beim Design des Moduls darauf geachtet, es weitgehend intuitiv und selbst-dokumentierend zu gestalten. Nichtsdestotrotz findet sich hier eine kurze
Einf�hrung in das \dijkstra - Modul.

\section{Funktions�bersicht}
Das Modul \dijkstra realisiert folgende Funktionen:
\begin{itemize}
	\item graphisches Erstellen / Bearbeiten eines Distanzgraphen
	\item Erstellen / Bearbeiten eines Graphen mittels Kanten- / Knotenliste oder Adjazenzmatrix
	\item Speichern und Laden von Graphen
	\item Visualisierung des Dijkstra-Algorithmus
\end{itemize}

\section{Modul starten}
Um das Modul zu starten, w�hlt man im Men� \textsc{<Datei>} das Submen� \textsc{<Neu>} und dann den Men�befehl \textsc{\dijkstra}. Im Hauptfenster erscheint nun die Oberfl�che des \dijkstra - Moduls im Eingabe-Modus.

\section{Symbolleiste}
Die Symbolleiste stellt die Funktionen \textsc{Speichern, Speichern unter, R�ckg�ngig} und \textsc{Wiederherstellen} bereit.
\newpage
\section{Programmstart - Der Willkommensbildschirm}
Nach Starten des Hauptprogramms \jalgo k�nnen Sie �ber den Button <\textsc{Neu}> oder mit dem Men�punkt \textsc{<Datei>$\rightarrow$<Neu>$\rightarrow$<\avl>} eine neue Instanz des Moduls \avl �ffnen. Anschlie�end �ffnet sich der Willkommensbildschirm des Moduls, der Ihnen verschiedene M�glichkeiten er�ffnet.\\
\bigskip
\centerpic{avl/welcomscreen}{0.5}{Der Willkommensbildschirm des Moduls \avl}
\bigskip

\subsectionicon{Baum laden}{avl/welcome_load}
Mit Klick auf das Ordner-Symbol �ffnet sich ein Dialogfenster, in dem Ihnen die M�glichkeit 
gegeben wird, eine \emph{"`*.jalgo"'} - Datei auszuw�hlen, in welcher ein Baum gespeichert wurde.\\
Im Prinzip ist die Bedeutung dieses Buttons die gleiche wie des <\textsc{�ffnen}>-Buttons in der Werkzeugleiste. Der Unterschied besteht darin, dass der Button in der Werkzeugleiste eine neue Modulinstanz �ffnet, in welcher die Datei geladen wird, der Button im Startbildschirm von \avl jedoch die Datei in die aktuell ge�ffnete Modulinstanz l�dt.

\subsectionicon{Baum von Hand erstellen}{avl/welcome_manual}
Mit Klick auf das Hand-Symbol gelangen Sie sofort zur leeren Arbeitsfl�che des Moduls \avl.
Sie k�nnen jetzt mit der knotenweisen Generierung eines neuen Suchbaumes beginnen.

\subsectionicon{Zufallsbaum erstellen lassen}{avl/welcome_random}
Mit Klick auf das W�rfel-Symbol beginnen Sie die Generierung eines zuf�llig erzeugten Suchbaumes. In dem folgenden Dialogfenster k�nnen Sie verschiedene Daten zum Baum und die Art der Visualisierung festlegen. 
\centerpic{avl/rgd}{1}{Eingabe der Zufallsbaumdaten}
\begin{itemize}
	\item {\bf Anzahl der Knoten}\\
	 Geben Sie hier die Anzahl der Knoten ein. Der entstehende Baum muss mindestens einen Knoten enthalten, h�chstens aber 99. 
	\item {\bf AVL-Eigenschaft}\\
	 Aktivieren Sie dieses K�stchen, wenn der zu erstellende Baum die AVL-Eigenschaft besitzen soll.
	\item {\bf Visualisierung}\\
	W�hlen Sie hier die Art der Visualisierung der Erstellung aus.
	\begin{itemize}
		\item {\sc keine}\\ Der Baum wird sofort erstellt.
		\item {\sc schrittweise}\\ Jeder Algorithmusschritt kann von Ihnen per Hand best�tigt werden.
		\item {\sc automatisch}\\ Lassen Sie die Erstellung des Baumes als Animation ablaufen, die Geschwindigkeit ist dabei einstellbar. 
	\end{itemize}
	Haben Sie schrittweise oder automatische Visualisierung gew�hlt, k�nnen Sie den Ablauf jederzeit abbrechen. 
    Dabei wird das gerade aktive Knoteneinf�gen abgebrochen, und der Baum steht mit entsprechend weniger Knoten zur Verf�gung.
\end{itemize}

\subsectionicon{Willkommensbildschirm anzeigen}{avl/logo}
Mit Klick auf diesen Button in der Werkzeugleiste des Modulbildschirms kann der Willkommensbildschirm sp�ter jederzeit wieder angezeigt werden. Dabei werden Sie eventuell gefragt, wie Sie mit Ihren �nderungen verfahren wollen. Sollten Sie Ihre �nderungen nicht verwerfen wollen, so wird eine neue Instanz des Moduls ge�ffnet.
\centerpic{avl/clearmessage}{1}{Dialog mit der Frage, ob der ganze Baum gel�scht werden soll.}

\bigskip
\section{Der Arbeitsbereich}
\label{sec:DerArbeitsbereich}

Der Arbeitsbereich ist untergliedert in drei Bereiche, die Ihnen Zugriff auf alle wesentlichen Funktionen in den Algorithmus-Phasen erm�glicht:

\bigskip
\centerpic{kmp/phase1screen}{0.5}{Phase 1 Bildschirm}
\bigskip

   
\begin{enumerate}
	\item  \textbf{Steuerung}\newline
      Hier wird der Algorithmus gestartet und gesteuert.
   \item \textbf{Visualisierung}\newline
      Hier werden die Verschiebetabelle bzw. der Text dargestellt.
   \item \textbf{Dokumentation}\newline
      Hier gibt es verschiedene Perspektiven, die miteinander kombiniert werden k�nnen: 			Quellcode, Text, Protokoll.
\end{enumerate}

\section{Anzeigeoptionen}
\label{sec:Anzeigeoptionen}

\subsection{Skalierung}
\label{sec:Skalierung}

Die Gr��e der Elemente des Visualisierungs- und Dokumentationsbereichs kann eingestellt werden, klicken Sie daf�r auf den Schieberegler im Visualisierungsbereich und ziehen Sie ihn nach oben bzw. unten.

\centerpic{kmp/slide}{0.5}{Auswirkung des Schiebereglers zur Skalierung}
\bigskip
\subsection{Aufteilung der Bereiche}
\label{sec:Aufteilung der Bereiche}

Die Aufteilung zwischen dem Visualisierungs- und dem Dokumentationsbereich kann mit einem Schiebebalken ver�ndert werden. Per Klick auf die Kante k�nnen Sie die Grenze verschieben.

\centerpic{kmp/move}{0.5}{Auswirkung des Schiebereglers}
\bigskip

\subsection{Beamer - Modus}
\label{sec:BeamerModus}

Der Beamer-Modus erm�glicht das schnelle Einstellen von Anzeigeoptionen, die die Pr�sentation in Vorlesungen oder �hnlichen Veranstaltungen beg�nstigen. Dieser Modus ist zu finden unter 'Knuth Morris Pratt' => 'Beamermodus'.
Die Anzeige ist f�r die Aufl�sung 1024x768 optimiert und vergr��ert die Elemente um den Faktor 1,5. Im Dokumentationsbereich wird die Perspektive 'Code' angezeigt.
Ist der Modus aktiv, so erscheint vor diesem Men�eintrag ein H�kchen. Um den Modus wieder auszuschalten, entfernen Sie einfach den Haken per Klick.

\centerpic{kmp/beamermod}{0.5}{Der Beamermodus}
\bigskip
\subsection{Zyklus - Anzeige}
\label{sec:ZyklusAnzeige}

Im Visualisierungsbereich der Phase 1 'Generierung der Verschiebetabelle' k�nnen optional die Zyklen angezeigt werden, setzen Sie dazu das H�kchen 'Zyklen' im Steuerungsbereich. Es werden maximal drei Zyklen gleichzeitig angezeigt.

\centerpic{kmp/cycle}{0.5}{Beispiel f�r Zyklen}
\bigskip
\section{Legende}
\label{sec:Legende}

\subsection{Elemente in der Phase 'Generierung der Verschiebetabelle'}
\label{sec:ElementeInDerPhaseGenerierungDerVerschiebetabelle}

\bigskip
\centerpic{kmp/phase1screen_ex1}{0.5}{Beispiel f�r Phase 1}
\bigskip
\textit{Visualisierungsbereich}\newline
\textbf{Pfeil schwarz mit wei�em 'P' }- Zeiger auf die Variable 'patpos', die Patternposition\newline
\textbf{Pfeil wei� mit schwarzem 'V' }- Zeiger auf die Variable 'VglInd', der Vergleichsindex\newline
\textbf{schwarzer Pfeil �ber der Tabelle }- die verglichenen Zeichen\newline
\textbf{gelber Hintergrund von Zellen} - Zellenkopf\newline
\textbf{schwarzer Rahmen um Zelle in der Zeile 'Index' }- aktuell kopierte Verschiebeinformation\newline
\textbf{roter Rahmen um Zellen }- negative boolesche Bedingung\newline
\textbf{lila Strich }- Zyklen, die im Pattern auftreten
\newline\newline
\textit{Schriftfarben}\newline
\textbf{blau} - aktuell geschriebene Verschiebeinformation (entspricht Zuweisungsstatement im Code)\newline
\textbf{rot} - die verglichenen Zeichen stimmen nicht �berein (entspricht booleschem Statement im Code)\newline
\textbf{gr�n} - die verglichenen Zeichen stimmen �berein (entspricht booleschem Statement im Code)

\subsection{Elemente in der Phase 'Suchen im Suchtext'}
\label{sec:ElementeInDerPhaseSuchenImSuchtext}
 \bigskip
\centerpic{kmp/phase2screen_ex1}{0.5}{Beispiel f�r diese Phase}
 \bigskip
\textit{Visualisierungsbereich}\newline
\textbf{lila Rahmen }- das Sichtfenster, welches die aktuell betrachteten Zeichen des Textes und des Patterns justiert
\newline\newline
\textit{Schriftfarben}\newline
\textbf{blau} - aktuell geschriebene Verschiebeinformation (entspricht Zuweisungsstatement im Code)

\subsection{Im Dokumentationsbereich}
\label{sec:ImDokumentationsbereich}

\bigskip
\centerpic{kmp/phase2screen_ex2}{0.5}{Beispiel f�r diese Phase}
\bigskip
\textit{Perspektive 'Code'}\newline
\textbf{roter Hintergrund }- negative boolesche Bedingung\newline
\textbf{gr�ner Hintergrund} - positive boolesche Bedingung\newline
\textbf{blauer Hintergrund }- sonstige Statements
\newline\newline
\textit{Perspektive 'Protokoll'}\newline
\textbf{blaue Schrift} - zuletzt vollzogener Schritt\newline
\textbf{schwarze Schrift} - f�r den aktuellen Stand irrelevante vorausgegangene Schritte
\newline\newline
\textit{Perspektive 'Suchtext'}\newline
\textbf{gelber Hintergrund} - der Textausschnitt, an dem das Pattern momentan anliegt

\section{Modulfunktionen}
\label{sec:Modulfunktionen}

\subsection{Eingabe von Pattern und Text}
\label{sec:EingabeVonPatternUndText}

\subsubsection{Pattern eingeben}
\label{sec:PatternEingeben}


\centerpic{kmp/pattern1}{0.5}{Pattern manuell eingeben}
\bigskip
Das Pattern kann manuell eingegeben werden, hierbei ist die maximale L�nge von 10 Zeichen zu beachten. Das Alphabet kann aus beliebigen Zeichen bestehen. Die Eingabe erfolgt im Steuerungsbereich in die Eingabezeile 'Pattern'. Um das Pattern zu �bernehmen klicken Sie auf 'Setzen'.

\subsubsection{Suchtext eingeben}


Der Text unterliegt keinen Beschr�nkungen und kann manuell eingegeben oder aus einer vorhandenen *.txt-Datei importiert werden.
Die Eingabe erfolgt im Steuerungsbereich, klicken Sie hierf�r auf 'Eingeben'.

\centerpic{kmp/text1}{0.5}{Text manuell eingeben}
\bigskip
Es �ffnet sich ein Fenster, in dem Sie den Text manuell eingeben (oder auch per Copy und Paste einf�gen) oder eine *.txt-Datei laden k�nnen. Um den Text zu �bernehmen klicken Sie auf 'Anwenden'.

\centerpic{kmp/text1}{0.5}{Text laden oder eingeben}
\bigskip
\subsection{Generieren von Pattern und passenden Texten}
\label{sec:GenerierenVonPatternUndPassendenTexten}

\subsubsection{Pattern eingeben}

\centerpic{kmp/pattern2}{0.5}{Pattern generieren lassen}
\bigskip
Neben der manuellen Eingabe des Patterns gibt es die M�glichkeit, ein Pattern generieren zu lassen. Klicken Sie daf�r auf den Knopf 'Zufall' (im Steuerungsbereich neben der Eingabezeile 'Pattern') und w�hlen Sie im folgenden Fenster die Kardinalit�t des zu nutzenden Alphabets und die gew�nschte L�nge des Patterns aus. Das Pattern wird mit Klick auf 'Anwenden' geniert und Sie kehren zum Arbeitsbereich zur�ck.

\centerpic{kmp/pattern3}{0.5}{Pattern generieren lassen}
\bigskip
\subsubsection{Suchtext eingeben}

\centerpic{kmp/text3}{0.5}{Suchtext generieren lassen}
\bigskip
Um einen zum Pattern passenden Text generieren zu lassen, klicken Sie auf 'Generieren' (im Steuerungsbereich neben 'Text'). Dadurch wird eine Text erstellt, der das Pattern enth�lt und nur aus den im Pattern genutzten Zeichen besteht. Dieser Text wird sofort �bernommen, Sie haben aber die M�glichkeit, ihn zu bearbeiten, indem Sie auf 'Eingeben' klicken.


\subsection{Generierung der Verschiebetabelle}
\label{sec:GenerierungDerVerschiebetabelle}


Wenn ein Pattern gesetzt wurde, kann die Verschiebetabelle erstellt werden. Um den Algorithmus zu starten und zu steuern, benutzen Sie die Pfeile im Steuerungsbereich.

\centerpic{kmp/control}{0.5}{Algorithmussteuerung}
\bigskip
Das Pattern kann nun noch erweitert werden ohne den aktuellen Algorithmus zu unterbrechen, geben Sie hierf�r das gew�nschte Zeichen in das Eingabefeld "Erweiterung" im Steuerungsbereich ein. Sollte das Pattern bereits die maximale L�nge haben, ist eine Erweiterung nicht m�glich.

\centerpic{kmp/expand}{0.5}{Pattern erweitern}
\bigskip
Nachdem die Verschiebetabelle erstellt wurde, k�nnen Pattern und Verschiebetabelle f�r eine Suche im Text genutzt werden. Klicken Sie hierf�r auf das Feld 'Weiter zu Phase 2', welches erst am Ende des Algorithmus im Steuerungsbereich erscheint.

\centerpic{kmp/go_p2}{0.5}{Von Phase 1 in Phase 2 wechseln}
\bigskip
\subsection{Suchen im Suchtext}
Erst wenn Pattern und Text gesetzt sind, kann das Anwenden der Verschiebetabelle an einem Text durchgef�hrt werden. Um Pattern und Text zu setzen, haben Sie verschiedene M�glichkeiten, siehe: Eingabe von Pattern und Text , Generieren von Pattern und passenden Texten.
Um den Algorithmus zu starten und zu steuern, benutzen Sie die Pfeile im Steuerungsbereich.

\centerpic{kmp/control}{0.5}{Algorithmussteuerung}
\bigskip
Sollten Sie w�hrend der Durchf�hrung des Algorithmus das Pattern und/oder den Text �ndern, wird die aktuelle Durchf�hrung des Algorithmus unterbrochen.
\bigskip
Der Suchvorgang wird beendet, wenn das Pattern gefunden wurde oder das Textende erreicht wurde, es erscheint eine Meldung dazu im Steuerungsbereich.

\centerpic{kmp/endnegativ}{0.5}{Nachricht, dass Pattern nicht gefunden wurde}
\bigskip
\centerpic{kmp/endpositiv}{0.5}{Nachricht, dass Pattern gefunden wurde}
\bigskip
\subsection{�ffnen einer KMP-Sitzung}
\label{sec:�ffnenEinerKMPSitzung}


Mit Klick auf das Ordner-Symbol �ffnet sich ein Dialogfenster, in dem Ihnen die M�glichkeit gegeben wird, eine '*.jalgo' - Datei auszuw�hlen, in welcher eine KMP-Sitzung gespeichert wurde.

\centerpic{kmp/load}{0.5}{Dialogfenster zum Laden einer Sitzung}
\bigskip

\subsection{Pr�sentation von Lernbeispielen}
Es stehen Ihnen zehn Beispiele zur Auswahl, die jeweils eine besondere Eigenschaft des KMP-Algorithmus repr�sentieren. Wenn Sie mit dem Mauszeiger �ber die Beispiele fahren, werden diese Eigenschaften kurz beschrieben. W�hlen Sie das gew�nschte Lernbeispiel aus und klicken Sie auf 'Laden' um das Beispiel zu starten.\newline
Die Beispiele bestehen aus Pattern- und passendem Text.

\centerpic{kmp/learning}{0.5}{Pr�sentation von Lernbeispielen}
\bigskip
Die Steuerung der Lernbeispiele erfolgt durch die Pfeile im Steuerungsbereich.


\section{Algorithmussteuerung}
Aufgabe des Moduls \avl ist es, Baumalgorithmen, wie das Einf�gen und L�schen von Knoten, zu visualisieren. Jeder Algorithmus ist in verschiedene Teilschritte unterteilt, die nacheinander angezeigt werden. Das Visualisieren erfolgt dabei durch das Zeichnen des Baumes, durch die Erkl�rung der Schritte im Dokumentationsbereich und im Logbuch und durch die Neuberechnung der baumspezifischen Daten, die im Infobereich pr�sentiert werden.\\
Nachdem Sie einen Algorithmus gestartet haben, verweilt er in einem Initialzustand und wartet auf Ihre Eingabe. Nun haben Sie die M�glichkeit, den Algorithmus in kleinen oder gro�en Schritten zu durchlaufen; Sie k�nnen ihn sofort beenden oder direkt abbrechen. Daf�r bietet die Algorithmussteuerung die entsprechenden Werkzeuge.

\subsection{Schritt-Pfeile}
Mittels der Schritt-Pfeile steuern Sie die Abfolge der Einzelschritte und bekommen so eine detaillierte Sicht auf die Arbeitsweise des Algorithmus. Das Programm bietet Ihnen die M�glichkeit, einen Teilschritt r�ckg�ngig zu machen und damit gewisse Abl�ufe zu wiederholen. Die Schritt-Pfeile, welche die R�ckg�ngigfunktion anbieten, weisen in ihrer Richtung nach links und sind dadurch intuitiv von den Vorw�rts-Pfeilen zu unterscheiden.\\
Zus�tzlich gibt es f�r jede Richtung einen gro�en und einen kleinen Schritt, der per Knopfdruck ausgef�hrt wird.

Kleine Schritte beim Einf�gen eines Knotens stellen Schl�sselvergleiche, Balancenberechnungen und Rotationen dar. Gro�e Schritte hingegen sind zum Beispiel das Suchen der Einf�gestelle, das Einf�gen an dieser und die gesamte Balancenaktualisierung.

\subsubsectiondoubleicon{Einzel-Schritt-Pfeile}{avl/icon_undo}{avl/icon_perform}
Ein Klick auf diese Buttons realisiert einen kleinen Algorithmusschritt zur�ck bzw. nach vorn.
\subsubsectiondoubleicon{Block-Schritt-Pfeile}{avl/icon_undo_blockstep}{avl/icon_perform_blockstep}
Ein Klick auf diese Buttons realisiert einen gro�en Schritt zur�ck bzw. nach vorn. Sollte der Algorithmusablauf an eine Stelle geraten, an der es nur noch einen kleinen Schritt nach vorn bzw. zur�ck gibt, so hat der Block-Schritt die selbe Funktionalit�t wie ein Einzel-Schritt.

\subsectiondoubleicon{Abbruch und Beenden-Buttons}{avl/icon_abort}{avl/icon_finish}
Klicken Sie auf den Beenden-Button \raisebox{-1ex}{\includegraphics[scale=0.8]{\pfad avl/icon_finish}} um den laufenden Algorithmus bis zum Ende auszuf�hren.\\
Klicken Sie auf den Abbruch-Button \raisebox{-1ex}{\includegraphics[scale=0.8]{\pfad avl/icon_abort}} um den laufenden Algorithmus abzubrechen. Der Baum hat danach den gleichen Status wie vor Beginn des Algorithmus.\\
Ist ein Algorithmus beendet, so steht Ihnen diese Option nicht mehr zur Verf�gung, weil nur der
\textit{laufende} Algorithmus abgebrochen werden kann.

\subsection{Animationsgeschwindigkeit}
Beim Generieren eines Zufallsbaumes haben Sie die Option, den Ablauf der Baumerzeugung als Animation ablaufen zu lassen. Starten Sie in diesem Modus, so beginnt die Animation sofort und kann mit dem Geschwindigkeitsregler schneller oder langsamer abgespielt werden. Zu Beginn steht dieser auf der mittleren Position. Verschieben Sie den Regler nach links, um die Animation zu verlangsamen bzw. nach rechts, um sie zu beschleunigen.\\
Eine Animation der anderen Algorithmenabl�ufe ist in dieser Version von \avl nicht integriert.

\bigskip
\centerpic{avl/animregler}{1}{Der Regler f�r die Animationsgeschwindigkeit}

\bigskip
\section{Dokumentation}
Da das Modul \avl vor allem zu Lehr- und Lernzwecken eingesetzt werden soll, ist eine detaillierte Dokumentation der Algorithmen unumg�nglich. F�r die Einzelheiten des Algorithmustextes steht ein Auszug aus dem Vorlesungsskript von Prof. Vogler zur Verf�gung. Ein Logbuch in der rechten unteren Ecke des Bildschirmes f�hrt Protokoll �ber den Stand und die Beschaffenheit des einzelnen Algorithmusteilschrittes.\\
Zu guter Letzt wird ein Infobereich angeboten, in dem wichtige Baumdaten zusammengefasst sind.

\subsection{Skript}
Der Dokumentationsbereich, der das Skript enth�lt, befindet sich am unteren Bildschirmrand. Es handelt sich hierbei um einen Auszug des Skripts zur Vorlesung "`Algorithmen und Datenstrukturen"' von Prof. Vogler (TU Dresden), Version vom 2. Oktober 2003. Im Rahmen dieser Vorlesung soll das Modul vorwiegend eingesetzt werden.\\
Bei dem jeweils aktuellen Algorithmustext handelt es sich um die Aktion, die als n�chstes im Ablauf des Algorithmus erfolgen wird. Sie wird rot markiert angezeigt.

\subsection{Logbuch}
Das Logbuch ist eine weitere M�glichkeit, den Ablauf des Algorithmus zu verfolgen. Es bezieht sich in erster Linie auf baumspezifische Daten und verwendet zum Beispiel konkrete Schl�sselwerte, anhand deren die Aktionen des Algorithmus besser verstanden werden sollen.\\
Auch hier wird der aktuelle Eintrag rot markiert dargestellt. Dieser bezieht sich aber auf die zuletzt ausgef�hrte Aktion.

\subsection{Infobereich}
Der Infobereich ist haupts�chlich daf�r gedacht, Ihnen schnell wichtige Daten des Baumes bereit zu stellen. Hier finden Sie folgende Punkte:
\begin{itemize}
	\item \textsc{Anzahl der Knoten}\\
		Dieser Punkt fasst f�r Sie die Anzahl der Knoten im Baum zusammen.
	\item \textsc{Baumh�he}\\
		Hier finden Sie die Anzahl der Level des Baumes.
	\item \textsc{Durchschnittliche Suchtiefe}\\
		Dieser Wert berechnet sich durch die Summe der Level aller Knoten geteilt durch die Anzahl dieser. Der Wert ist ein Indiz daf�r, wie gut der Baum ausbalanciert ist bzw. wie gro� der Suchaufwand im Durchschnitt ist.
\end{itemize}

\bigskip
\section{Zusatzfunktionen}
Dieses Kapitel widmet sich den Eastereggs des Moduls \avl.\\
Sollten Sie sich lieber selber gerne auf die Suche nach diesen Zusatzfunktionen machen wollen, so �berspringen Sie besser dieses Kapitel.\\
F�r alle Anderen folgt nun eine �bersicht zum Baumnavigator und dem Beamermodus.

\subsection{Navigator}
Der Navigator ist eine kleine, versteckte Zusatzfunktion, die die Arbeit mit gro�en B�umen erheblich vereinfachen kann. Er stellt eine willkommene Hilfe f�r das Scrollen der Zeichenfl�che dar, ist aber nicht so einfach zu finden.
\begin{itemize}
	\item Wenn der Baum, der auf der Zeichenfl�che angezeigt wird, zu gro� f�r diese wird, so erscheinen Schiebebalken, mit denen Sie den Bildausschnitt verschieben k�nnen.
	\item Klicken Sie nun auf das kleine Quadrat in der rechten unteren Ecke der Zeichenfl�che, genau zwischen den beiden Schiebebalken. Halten Sie dabei die linke Maustaste gedr�ckt.
	\item Eine kleine �bersichtskarte des Baumes mit einem Ausschnittfenster erscheint. Bewegen Sie die Maus (mit gedr�ckter Taste) und das Ausschnittfenster, das den Bildschirminhalt der Zeichenfl�che repr�sentiert, folgt Ihren Bewegungen.
\end{itemize}
\bigskip
\begin{center}
	\includegraphics[scale=0.7]{\pfad avl/navigator1} \hfill
	\includegraphics[scale=0.7]{\pfad avl/navigator2} \\
	Ein Klick auf das kleine K�stchen... \hfill ...�ffnet den Navigator!
\end{center}

\subsectionicon{Beamermodus}{avl/icon_beamer}
Der Beamermodus ist in erster Linie f�r die Pr�sentation in Vorlesungen oder �hnlichen Veranstaltungen gedacht. Ist dieser Modus aktiv, so werden die Knoten des Baumes und die Eintr�ge des Logbuches vergr��ert dargestellt. Der Algorithmustext aus dem Skript von Prof. Vogler bleibt dabei unver�ndert, weil davon ausgegangen wird, dass die interessierten Studenten der Vorlesung �ber ein (eventuell aktuelleres) Skript verf�gen.\\
Sie erreichen den Beamermodus �ber den Men�punkt \textsc{<\avl>} $\rightarrow$ \textsc{<Beamermodus>}. Ist der Modus aktiv, so erscheint neben diesem Men�eintrag ein H�kchen. Um den Modus wieder auszuschalten, entfernen Sie einfach den Haken per Klick.
\bigskip
\centerpic{avl/beamermenu}{0.5}{Das Men� \textsc{<\avl>} mit dem Eintrag \textsc{<Beamermodus>}}

\newpage
\section{Impressum}
Die \jalgo Software wurde im Sommersemester 2004 von der Praktikumsgruppe SWT04-PROG1 im Rahmen des externen Softwarepraktikums entwickelt. Mitwirkende waren die
\paragraph{Teammitglieder}
\begin{itemize}
	\item Michael Pradel --- Chief of Algorithms
	\item Cornelius Hald --- Chief of Framework
	\item Malte Blumberg
	\item Stephan Creutz
	\item Christopher Friedrich
	\item Anne Kersten
	\item Hauke Menges
	\item Babett Schalitz
	\item Benjamin Scholz
	\item Marco Zimmerling
\end{itemize}
Die Webseite des Praktikums finden Sie unter \url{http://web.inf.tu-dresden.de/~swt04-p1/}.\\
Komplette �berarbeitung erfuhr die Software und die Dokumentation unter anderem durch Alexander Claus und Matthias Schmidt. Weitergehende Informationen �ber \jalgo erhalten Sie unter \url{http://j-algo.binaervarianz.de/}.
\newcommand{\avl}{\mbox{\bfseries AVL-B�ume} }

\chapter{Das Modul Dijkstra}
\section{Einleitung}
Das Modul \dijkstra visualisiert den bekannten Algorithmus von E. W. Dijkstra zum Finden der k�rzesten Wege von einem Startknoten in einem Distanzgraphen. Der Algorithmus selbst ist unter anderem im Vorlesungsskript von Prof. Vogler "`Algorithmen, Datenstrukturen und Programmierung"' zu finden. Aber auch im Internet existieren zahlreiche Quellen dazu.

Soweit es m�glich gewesen ist, wurde beim Design des Moduls darauf geachtet, es weitgehend intuitiv und selbst-dokumentierend zu gestalten. Nichtsdestotrotz findet sich hier eine kurze
Einf�hrung in das \dijkstra - Modul.

\section{Funktions�bersicht}
Das Modul \dijkstra realisiert folgende Funktionen:
\begin{itemize}
	\item graphisches Erstellen / Bearbeiten eines Distanzgraphen
	\item Erstellen / Bearbeiten eines Graphen mittels Kanten- / Knotenliste oder Adjazenzmatrix
	\item Speichern und Laden von Graphen
	\item Visualisierung des Dijkstra-Algorithmus
\end{itemize}

\section{Modul starten}
Um das Modul zu starten, w�hlt man im Men� \textsc{<Datei>} das Submen� \textsc{<Neu>} und dann den Men�befehl \textsc{\dijkstra}. Im Hauptfenster erscheint nun die Oberfl�che des \dijkstra - Moduls im Eingabe-Modus.

\section{Symbolleiste}
Die Symbolleiste stellt die Funktionen \textsc{Speichern, Speichern unter, R�ckg�ngig} und \textsc{Wiederherstellen} bereit.
\newpage
\section{Programmstart - Der Willkommensbildschirm}
Nach Starten des Hauptprogramms \jalgo k�nnen Sie �ber den Button <\textsc{Neu}> oder mit dem Men�punkt \textsc{<Datei>$\rightarrow$<Neu>$\rightarrow$<\avl>} eine neue Instanz des Moduls \avl �ffnen. Anschlie�end �ffnet sich der Willkommensbildschirm des Moduls, der Ihnen verschiedene M�glichkeiten er�ffnet.\\
\bigskip
\centerpic{avl/welcomscreen}{0.5}{Der Willkommensbildschirm des Moduls \avl}
\bigskip

\subsectionicon{Baum laden}{avl/welcome_load}
Mit Klick auf das Ordner-Symbol �ffnet sich ein Dialogfenster, in dem Ihnen die M�glichkeit 
gegeben wird, eine \emph{"`*.jalgo"'} - Datei auszuw�hlen, in welcher ein Baum gespeichert wurde.\\
Im Prinzip ist die Bedeutung dieses Buttons die gleiche wie des <\textsc{�ffnen}>-Buttons in der Werkzeugleiste. Der Unterschied besteht darin, dass der Button in der Werkzeugleiste eine neue Modulinstanz �ffnet, in welcher die Datei geladen wird, der Button im Startbildschirm von \avl jedoch die Datei in die aktuell ge�ffnete Modulinstanz l�dt.

\subsectionicon{Baum von Hand erstellen}{avl/welcome_manual}
Mit Klick auf das Hand-Symbol gelangen Sie sofort zur leeren Arbeitsfl�che des Moduls \avl.
Sie k�nnen jetzt mit der knotenweisen Generierung eines neuen Suchbaumes beginnen.

\subsectionicon{Zufallsbaum erstellen lassen}{avl/welcome_random}
Mit Klick auf das W�rfel-Symbol beginnen Sie die Generierung eines zuf�llig erzeugten Suchbaumes. In dem folgenden Dialogfenster k�nnen Sie verschiedene Daten zum Baum und die Art der Visualisierung festlegen. 
\centerpic{avl/rgd}{1}{Eingabe der Zufallsbaumdaten}
\begin{itemize}
	\item {\bf Anzahl der Knoten}\\
	 Geben Sie hier die Anzahl der Knoten ein. Der entstehende Baum muss mindestens einen Knoten enthalten, h�chstens aber 99. 
	\item {\bf AVL-Eigenschaft}\\
	 Aktivieren Sie dieses K�stchen, wenn der zu erstellende Baum die AVL-Eigenschaft besitzen soll.
	\item {\bf Visualisierung}\\
	W�hlen Sie hier die Art der Visualisierung der Erstellung aus.
	\begin{itemize}
		\item {\sc keine}\\ Der Baum wird sofort erstellt.
		\item {\sc schrittweise}\\ Jeder Algorithmusschritt kann von Ihnen per Hand best�tigt werden.
		\item {\sc automatisch}\\ Lassen Sie die Erstellung des Baumes als Animation ablaufen, die Geschwindigkeit ist dabei einstellbar. 
	\end{itemize}
	Haben Sie schrittweise oder automatische Visualisierung gew�hlt, k�nnen Sie den Ablauf jederzeit abbrechen. 
    Dabei wird das gerade aktive Knoteneinf�gen abgebrochen, und der Baum steht mit entsprechend weniger Knoten zur Verf�gung.
\end{itemize}

\subsectionicon{Willkommensbildschirm anzeigen}{avl/logo}
Mit Klick auf diesen Button in der Werkzeugleiste des Modulbildschirms kann der Willkommensbildschirm sp�ter jederzeit wieder angezeigt werden. Dabei werden Sie eventuell gefragt, wie Sie mit Ihren �nderungen verfahren wollen. Sollten Sie Ihre �nderungen nicht verwerfen wollen, so wird eine neue Instanz des Moduls ge�ffnet.
\centerpic{avl/clearmessage}{1}{Dialog mit der Frage, ob der ganze Baum gel�scht werden soll.}

\bigskip
\section{Der Arbeitsbereich}
\label{sec:DerArbeitsbereich}

Der Arbeitsbereich ist untergliedert in drei Bereiche, die Ihnen Zugriff auf alle wesentlichen Funktionen in den Algorithmus-Phasen erm�glicht:

\bigskip
\centerpic{kmp/phase1screen}{0.5}{Phase 1 Bildschirm}
\bigskip

   
\begin{enumerate}
	\item  \textbf{Steuerung}\newline
      Hier wird der Algorithmus gestartet und gesteuert.
   \item \textbf{Visualisierung}\newline
      Hier werden die Verschiebetabelle bzw. der Text dargestellt.
   \item \textbf{Dokumentation}\newline
      Hier gibt es verschiedene Perspektiven, die miteinander kombiniert werden k�nnen: 			Quellcode, Text, Protokoll.
\end{enumerate}

\section{Anzeigeoptionen}
\label{sec:Anzeigeoptionen}

\subsection{Skalierung}
\label{sec:Skalierung}

Die Gr��e der Elemente des Visualisierungs- und Dokumentationsbereichs kann eingestellt werden, klicken Sie daf�r auf den Schieberegler im Visualisierungsbereich und ziehen Sie ihn nach oben bzw. unten.

\centerpic{kmp/slide}{0.5}{Auswirkung des Schiebereglers zur Skalierung}
\bigskip
\subsection{Aufteilung der Bereiche}
\label{sec:Aufteilung der Bereiche}

Die Aufteilung zwischen dem Visualisierungs- und dem Dokumentationsbereich kann mit einem Schiebebalken ver�ndert werden. Per Klick auf die Kante k�nnen Sie die Grenze verschieben.

\centerpic{kmp/move}{0.5}{Auswirkung des Schiebereglers}
\bigskip

\subsection{Beamer - Modus}
\label{sec:BeamerModus}

Der Beamer-Modus erm�glicht das schnelle Einstellen von Anzeigeoptionen, die die Pr�sentation in Vorlesungen oder �hnlichen Veranstaltungen beg�nstigen. Dieser Modus ist zu finden unter 'Knuth Morris Pratt' => 'Beamermodus'.
Die Anzeige ist f�r die Aufl�sung 1024x768 optimiert und vergr��ert die Elemente um den Faktor 1,5. Im Dokumentationsbereich wird die Perspektive 'Code' angezeigt.
Ist der Modus aktiv, so erscheint vor diesem Men�eintrag ein H�kchen. Um den Modus wieder auszuschalten, entfernen Sie einfach den Haken per Klick.

\centerpic{kmp/beamermod}{0.5}{Der Beamermodus}
\bigskip
\subsection{Zyklus - Anzeige}
\label{sec:ZyklusAnzeige}

Im Visualisierungsbereich der Phase 1 'Generierung der Verschiebetabelle' k�nnen optional die Zyklen angezeigt werden, setzen Sie dazu das H�kchen 'Zyklen' im Steuerungsbereich. Es werden maximal drei Zyklen gleichzeitig angezeigt.

\centerpic{kmp/cycle}{0.5}{Beispiel f�r Zyklen}
\bigskip
\section{Legende}
\label{sec:Legende}

\subsection{Elemente in der Phase 'Generierung der Verschiebetabelle'}
\label{sec:ElementeInDerPhaseGenerierungDerVerschiebetabelle}

\bigskip
\centerpic{kmp/phase1screen_ex1}{0.5}{Beispiel f�r Phase 1}
\bigskip
\textit{Visualisierungsbereich}\newline
\textbf{Pfeil schwarz mit wei�em 'P' }- Zeiger auf die Variable 'patpos', die Patternposition\newline
\textbf{Pfeil wei� mit schwarzem 'V' }- Zeiger auf die Variable 'VglInd', der Vergleichsindex\newline
\textbf{schwarzer Pfeil �ber der Tabelle }- die verglichenen Zeichen\newline
\textbf{gelber Hintergrund von Zellen} - Zellenkopf\newline
\textbf{schwarzer Rahmen um Zelle in der Zeile 'Index' }- aktuell kopierte Verschiebeinformation\newline
\textbf{roter Rahmen um Zellen }- negative boolesche Bedingung\newline
\textbf{lila Strich }- Zyklen, die im Pattern auftreten
\newline\newline
\textit{Schriftfarben}\newline
\textbf{blau} - aktuell geschriebene Verschiebeinformation (entspricht Zuweisungsstatement im Code)\newline
\textbf{rot} - die verglichenen Zeichen stimmen nicht �berein (entspricht booleschem Statement im Code)\newline
\textbf{gr�n} - die verglichenen Zeichen stimmen �berein (entspricht booleschem Statement im Code)

\subsection{Elemente in der Phase 'Suchen im Suchtext'}
\label{sec:ElementeInDerPhaseSuchenImSuchtext}
 \bigskip
\centerpic{kmp/phase2screen_ex1}{0.5}{Beispiel f�r diese Phase}
 \bigskip
\textit{Visualisierungsbereich}\newline
\textbf{lila Rahmen }- das Sichtfenster, welches die aktuell betrachteten Zeichen des Textes und des Patterns justiert
\newline\newline
\textit{Schriftfarben}\newline
\textbf{blau} - aktuell geschriebene Verschiebeinformation (entspricht Zuweisungsstatement im Code)

\subsection{Im Dokumentationsbereich}
\label{sec:ImDokumentationsbereich}

\bigskip
\centerpic{kmp/phase2screen_ex2}{0.5}{Beispiel f�r diese Phase}
\bigskip
\textit{Perspektive 'Code'}\newline
\textbf{roter Hintergrund }- negative boolesche Bedingung\newline
\textbf{gr�ner Hintergrund} - positive boolesche Bedingung\newline
\textbf{blauer Hintergrund }- sonstige Statements
\newline\newline
\textit{Perspektive 'Protokoll'}\newline
\textbf{blaue Schrift} - zuletzt vollzogener Schritt\newline
\textbf{schwarze Schrift} - f�r den aktuellen Stand irrelevante vorausgegangene Schritte
\newline\newline
\textit{Perspektive 'Suchtext'}\newline
\textbf{gelber Hintergrund} - der Textausschnitt, an dem das Pattern momentan anliegt

\section{Modulfunktionen}
\label{sec:Modulfunktionen}

\subsection{Eingabe von Pattern und Text}
\label{sec:EingabeVonPatternUndText}

\subsubsection{Pattern eingeben}
\label{sec:PatternEingeben}


\centerpic{kmp/pattern1}{0.5}{Pattern manuell eingeben}
\bigskip
Das Pattern kann manuell eingegeben werden, hierbei ist die maximale L�nge von 10 Zeichen zu beachten. Das Alphabet kann aus beliebigen Zeichen bestehen. Die Eingabe erfolgt im Steuerungsbereich in die Eingabezeile 'Pattern'. Um das Pattern zu �bernehmen klicken Sie auf 'Setzen'.

\subsubsection{Suchtext eingeben}


Der Text unterliegt keinen Beschr�nkungen und kann manuell eingegeben oder aus einer vorhandenen *.txt-Datei importiert werden.
Die Eingabe erfolgt im Steuerungsbereich, klicken Sie hierf�r auf 'Eingeben'.

\centerpic{kmp/text1}{0.5}{Text manuell eingeben}
\bigskip
Es �ffnet sich ein Fenster, in dem Sie den Text manuell eingeben (oder auch per Copy und Paste einf�gen) oder eine *.txt-Datei laden k�nnen. Um den Text zu �bernehmen klicken Sie auf 'Anwenden'.

\centerpic{kmp/text1}{0.5}{Text laden oder eingeben}
\bigskip
\subsection{Generieren von Pattern und passenden Texten}
\label{sec:GenerierenVonPatternUndPassendenTexten}

\subsubsection{Pattern eingeben}

\centerpic{kmp/pattern2}{0.5}{Pattern generieren lassen}
\bigskip
Neben der manuellen Eingabe des Patterns gibt es die M�glichkeit, ein Pattern generieren zu lassen. Klicken Sie daf�r auf den Knopf 'Zufall' (im Steuerungsbereich neben der Eingabezeile 'Pattern') und w�hlen Sie im folgenden Fenster die Kardinalit�t des zu nutzenden Alphabets und die gew�nschte L�nge des Patterns aus. Das Pattern wird mit Klick auf 'Anwenden' geniert und Sie kehren zum Arbeitsbereich zur�ck.

\centerpic{kmp/pattern3}{0.5}{Pattern generieren lassen}
\bigskip
\subsubsection{Suchtext eingeben}

\centerpic{kmp/text3}{0.5}{Suchtext generieren lassen}
\bigskip
Um einen zum Pattern passenden Text generieren zu lassen, klicken Sie auf 'Generieren' (im Steuerungsbereich neben 'Text'). Dadurch wird eine Text erstellt, der das Pattern enth�lt und nur aus den im Pattern genutzten Zeichen besteht. Dieser Text wird sofort �bernommen, Sie haben aber die M�glichkeit, ihn zu bearbeiten, indem Sie auf 'Eingeben' klicken.


\subsection{Generierung der Verschiebetabelle}
\label{sec:GenerierungDerVerschiebetabelle}


Wenn ein Pattern gesetzt wurde, kann die Verschiebetabelle erstellt werden. Um den Algorithmus zu starten und zu steuern, benutzen Sie die Pfeile im Steuerungsbereich.

\centerpic{kmp/control}{0.5}{Algorithmussteuerung}
\bigskip
Das Pattern kann nun noch erweitert werden ohne den aktuellen Algorithmus zu unterbrechen, geben Sie hierf�r das gew�nschte Zeichen in das Eingabefeld "Erweiterung" im Steuerungsbereich ein. Sollte das Pattern bereits die maximale L�nge haben, ist eine Erweiterung nicht m�glich.

\centerpic{kmp/expand}{0.5}{Pattern erweitern}
\bigskip
Nachdem die Verschiebetabelle erstellt wurde, k�nnen Pattern und Verschiebetabelle f�r eine Suche im Text genutzt werden. Klicken Sie hierf�r auf das Feld 'Weiter zu Phase 2', welches erst am Ende des Algorithmus im Steuerungsbereich erscheint.

\centerpic{kmp/go_p2}{0.5}{Von Phase 1 in Phase 2 wechseln}
\bigskip
\subsection{Suchen im Suchtext}
Erst wenn Pattern und Text gesetzt sind, kann das Anwenden der Verschiebetabelle an einem Text durchgef�hrt werden. Um Pattern und Text zu setzen, haben Sie verschiedene M�glichkeiten, siehe: Eingabe von Pattern und Text , Generieren von Pattern und passenden Texten.
Um den Algorithmus zu starten und zu steuern, benutzen Sie die Pfeile im Steuerungsbereich.

\centerpic{kmp/control}{0.5}{Algorithmussteuerung}
\bigskip
Sollten Sie w�hrend der Durchf�hrung des Algorithmus das Pattern und/oder den Text �ndern, wird die aktuelle Durchf�hrung des Algorithmus unterbrochen.
\bigskip
Der Suchvorgang wird beendet, wenn das Pattern gefunden wurde oder das Textende erreicht wurde, es erscheint eine Meldung dazu im Steuerungsbereich.

\centerpic{kmp/endnegativ}{0.5}{Nachricht, dass Pattern nicht gefunden wurde}
\bigskip
\centerpic{kmp/endpositiv}{0.5}{Nachricht, dass Pattern gefunden wurde}
\bigskip
\subsection{�ffnen einer KMP-Sitzung}
\label{sec:�ffnenEinerKMPSitzung}


Mit Klick auf das Ordner-Symbol �ffnet sich ein Dialogfenster, in dem Ihnen die M�glichkeit gegeben wird, eine '*.jalgo' - Datei auszuw�hlen, in welcher eine KMP-Sitzung gespeichert wurde.

\centerpic{kmp/load}{0.5}{Dialogfenster zum Laden einer Sitzung}
\bigskip

\subsection{Pr�sentation von Lernbeispielen}
Es stehen Ihnen zehn Beispiele zur Auswahl, die jeweils eine besondere Eigenschaft des KMP-Algorithmus repr�sentieren. Wenn Sie mit dem Mauszeiger �ber die Beispiele fahren, werden diese Eigenschaften kurz beschrieben. W�hlen Sie das gew�nschte Lernbeispiel aus und klicken Sie auf 'Laden' um das Beispiel zu starten.\newline
Die Beispiele bestehen aus Pattern- und passendem Text.

\centerpic{kmp/learning}{0.5}{Pr�sentation von Lernbeispielen}
\bigskip
Die Steuerung der Lernbeispiele erfolgt durch die Pfeile im Steuerungsbereich.


\section{Algorithmussteuerung}
Aufgabe des Moduls \avl ist es, Baumalgorithmen, wie das Einf�gen und L�schen von Knoten, zu visualisieren. Jeder Algorithmus ist in verschiedene Teilschritte unterteilt, die nacheinander angezeigt werden. Das Visualisieren erfolgt dabei durch das Zeichnen des Baumes, durch die Erkl�rung der Schritte im Dokumentationsbereich und im Logbuch und durch die Neuberechnung der baumspezifischen Daten, die im Infobereich pr�sentiert werden.\\
Nachdem Sie einen Algorithmus gestartet haben, verweilt er in einem Initialzustand und wartet auf Ihre Eingabe. Nun haben Sie die M�glichkeit, den Algorithmus in kleinen oder gro�en Schritten zu durchlaufen; Sie k�nnen ihn sofort beenden oder direkt abbrechen. Daf�r bietet die Algorithmussteuerung die entsprechenden Werkzeuge.

\subsection{Schritt-Pfeile}
Mittels der Schritt-Pfeile steuern Sie die Abfolge der Einzelschritte und bekommen so eine detaillierte Sicht auf die Arbeitsweise des Algorithmus. Das Programm bietet Ihnen die M�glichkeit, einen Teilschritt r�ckg�ngig zu machen und damit gewisse Abl�ufe zu wiederholen. Die Schritt-Pfeile, welche die R�ckg�ngigfunktion anbieten, weisen in ihrer Richtung nach links und sind dadurch intuitiv von den Vorw�rts-Pfeilen zu unterscheiden.\\
Zus�tzlich gibt es f�r jede Richtung einen gro�en und einen kleinen Schritt, der per Knopfdruck ausgef�hrt wird.

Kleine Schritte beim Einf�gen eines Knotens stellen Schl�sselvergleiche, Balancenberechnungen und Rotationen dar. Gro�e Schritte hingegen sind zum Beispiel das Suchen der Einf�gestelle, das Einf�gen an dieser und die gesamte Balancenaktualisierung.

\subsubsectiondoubleicon{Einzel-Schritt-Pfeile}{avl/icon_undo}{avl/icon_perform}
Ein Klick auf diese Buttons realisiert einen kleinen Algorithmusschritt zur�ck bzw. nach vorn.
\subsubsectiondoubleicon{Block-Schritt-Pfeile}{avl/icon_undo_blockstep}{avl/icon_perform_blockstep}
Ein Klick auf diese Buttons realisiert einen gro�en Schritt zur�ck bzw. nach vorn. Sollte der Algorithmusablauf an eine Stelle geraten, an der es nur noch einen kleinen Schritt nach vorn bzw. zur�ck gibt, so hat der Block-Schritt die selbe Funktionalit�t wie ein Einzel-Schritt.

\subsectiondoubleicon{Abbruch und Beenden-Buttons}{avl/icon_abort}{avl/icon_finish}
Klicken Sie auf den Beenden-Button \raisebox{-1ex}{\includegraphics[scale=0.8]{\pfad avl/icon_finish}} um den laufenden Algorithmus bis zum Ende auszuf�hren.\\
Klicken Sie auf den Abbruch-Button \raisebox{-1ex}{\includegraphics[scale=0.8]{\pfad avl/icon_abort}} um den laufenden Algorithmus abzubrechen. Der Baum hat danach den gleichen Status wie vor Beginn des Algorithmus.\\
Ist ein Algorithmus beendet, so steht Ihnen diese Option nicht mehr zur Verf�gung, weil nur der
\textit{laufende} Algorithmus abgebrochen werden kann.

\subsection{Animationsgeschwindigkeit}
Beim Generieren eines Zufallsbaumes haben Sie die Option, den Ablauf der Baumerzeugung als Animation ablaufen zu lassen. Starten Sie in diesem Modus, so beginnt die Animation sofort und kann mit dem Geschwindigkeitsregler schneller oder langsamer abgespielt werden. Zu Beginn steht dieser auf der mittleren Position. Verschieben Sie den Regler nach links, um die Animation zu verlangsamen bzw. nach rechts, um sie zu beschleunigen.\\
Eine Animation der anderen Algorithmenabl�ufe ist in dieser Version von \avl nicht integriert.

\bigskip
\centerpic{avl/animregler}{1}{Der Regler f�r die Animationsgeschwindigkeit}

\bigskip
\section{Dokumentation}
Da das Modul \avl vor allem zu Lehr- und Lernzwecken eingesetzt werden soll, ist eine detaillierte Dokumentation der Algorithmen unumg�nglich. F�r die Einzelheiten des Algorithmustextes steht ein Auszug aus dem Vorlesungsskript von Prof. Vogler zur Verf�gung. Ein Logbuch in der rechten unteren Ecke des Bildschirmes f�hrt Protokoll �ber den Stand und die Beschaffenheit des einzelnen Algorithmusteilschrittes.\\
Zu guter Letzt wird ein Infobereich angeboten, in dem wichtige Baumdaten zusammengefasst sind.

\subsection{Skript}
Der Dokumentationsbereich, der das Skript enth�lt, befindet sich am unteren Bildschirmrand. Es handelt sich hierbei um einen Auszug des Skripts zur Vorlesung "`Algorithmen und Datenstrukturen"' von Prof. Vogler (TU Dresden), Version vom 2. Oktober 2003. Im Rahmen dieser Vorlesung soll das Modul vorwiegend eingesetzt werden.\\
Bei dem jeweils aktuellen Algorithmustext handelt es sich um die Aktion, die als n�chstes im Ablauf des Algorithmus erfolgen wird. Sie wird rot markiert angezeigt.

\subsection{Logbuch}
Das Logbuch ist eine weitere M�glichkeit, den Ablauf des Algorithmus zu verfolgen. Es bezieht sich in erster Linie auf baumspezifische Daten und verwendet zum Beispiel konkrete Schl�sselwerte, anhand deren die Aktionen des Algorithmus besser verstanden werden sollen.\\
Auch hier wird der aktuelle Eintrag rot markiert dargestellt. Dieser bezieht sich aber auf die zuletzt ausgef�hrte Aktion.

\subsection{Infobereich}
Der Infobereich ist haupts�chlich daf�r gedacht, Ihnen schnell wichtige Daten des Baumes bereit zu stellen. Hier finden Sie folgende Punkte:
\begin{itemize}
	\item \textsc{Anzahl der Knoten}\\
		Dieser Punkt fasst f�r Sie die Anzahl der Knoten im Baum zusammen.
	\item \textsc{Baumh�he}\\
		Hier finden Sie die Anzahl der Level des Baumes.
	\item \textsc{Durchschnittliche Suchtiefe}\\
		Dieser Wert berechnet sich durch die Summe der Level aller Knoten geteilt durch die Anzahl dieser. Der Wert ist ein Indiz daf�r, wie gut der Baum ausbalanciert ist bzw. wie gro� der Suchaufwand im Durchschnitt ist.
\end{itemize}

\bigskip
\section{Zusatzfunktionen}
Dieses Kapitel widmet sich den Eastereggs des Moduls \avl.\\
Sollten Sie sich lieber selber gerne auf die Suche nach diesen Zusatzfunktionen machen wollen, so �berspringen Sie besser dieses Kapitel.\\
F�r alle Anderen folgt nun eine �bersicht zum Baumnavigator und dem Beamermodus.

\subsection{Navigator}
Der Navigator ist eine kleine, versteckte Zusatzfunktion, die die Arbeit mit gro�en B�umen erheblich vereinfachen kann. Er stellt eine willkommene Hilfe f�r das Scrollen der Zeichenfl�che dar, ist aber nicht so einfach zu finden.
\begin{itemize}
	\item Wenn der Baum, der auf der Zeichenfl�che angezeigt wird, zu gro� f�r diese wird, so erscheinen Schiebebalken, mit denen Sie den Bildausschnitt verschieben k�nnen.
	\item Klicken Sie nun auf das kleine Quadrat in der rechten unteren Ecke der Zeichenfl�che, genau zwischen den beiden Schiebebalken. Halten Sie dabei die linke Maustaste gedr�ckt.
	\item Eine kleine �bersichtskarte des Baumes mit einem Ausschnittfenster erscheint. Bewegen Sie die Maus (mit gedr�ckter Taste) und das Ausschnittfenster, das den Bildschirminhalt der Zeichenfl�che repr�sentiert, folgt Ihren Bewegungen.
\end{itemize}
\bigskip
\begin{center}
	\includegraphics[scale=0.7]{\pfad avl/navigator1} \hfill
	\includegraphics[scale=0.7]{\pfad avl/navigator2} \\
	Ein Klick auf das kleine K�stchen... \hfill ...�ffnet den Navigator!
\end{center}

\subsectionicon{Beamermodus}{avl/icon_beamer}
Der Beamermodus ist in erster Linie f�r die Pr�sentation in Vorlesungen oder �hnlichen Veranstaltungen gedacht. Ist dieser Modus aktiv, so werden die Knoten des Baumes und die Eintr�ge des Logbuches vergr��ert dargestellt. Der Algorithmustext aus dem Skript von Prof. Vogler bleibt dabei unver�ndert, weil davon ausgegangen wird, dass die interessierten Studenten der Vorlesung �ber ein (eventuell aktuelleres) Skript verf�gen.\\
Sie erreichen den Beamermodus �ber den Men�punkt \textsc{<\avl>} $\rightarrow$ \textsc{<Beamermodus>}. Ist der Modus aktiv, so erscheint neben diesem Men�eintrag ein H�kchen. Um den Modus wieder auszuschalten, entfernen Sie einfach den Haken per Klick.
\bigskip
\centerpic{avl/beamermenu}{0.5}{Das Men� \textsc{<\avl>} mit dem Eintrag \textsc{<Beamermodus>}}

\newpage
\section{Impressum}
Die \jalgo Software wurde im Sommersemester 2004 von der Praktikumsgruppe SWT04-PROG1 im Rahmen des externen Softwarepraktikums entwickelt. Mitwirkende waren die
\paragraph{Teammitglieder}
\begin{itemize}
	\item Michael Pradel --- Chief of Algorithms
	\item Cornelius Hald --- Chief of Framework
	\item Malte Blumberg
	\item Stephan Creutz
	\item Christopher Friedrich
	\item Anne Kersten
	\item Hauke Menges
	\item Babett Schalitz
	\item Benjamin Scholz
	\item Marco Zimmerling
\end{itemize}
Die Webseite des Praktikums finden Sie unter \url{http://web.inf.tu-dresden.de/~swt04-p1/}.\\
Komplette �berarbeitung erfuhr die Software und die Dokumentation unter anderem durch Alexander Claus und Matthias Schmidt. Weitergehende Informationen �ber \jalgo erhalten Sie unter \url{http://j-algo.binaervarianz.de/}.
\newcommand{\avl}{\mbox{\bfseries AVL-B�ume} }

\chapter{Das Modul Dijkstra}
\section{Einleitung}
Das Modul \dijkstra visualisiert den bekannten Algorithmus von E. W. Dijkstra zum Finden der k�rzesten Wege von einem Startknoten in einem Distanzgraphen. Der Algorithmus selbst ist unter anderem im Vorlesungsskript von Prof. Vogler "`Algorithmen, Datenstrukturen und Programmierung"' zu finden. Aber auch im Internet existieren zahlreiche Quellen dazu.

Soweit es m�glich gewesen ist, wurde beim Design des Moduls darauf geachtet, es weitgehend intuitiv und selbst-dokumentierend zu gestalten. Nichtsdestotrotz findet sich hier eine kurze
Einf�hrung in das \dijkstra - Modul.

\section{Funktions�bersicht}
Das Modul \dijkstra realisiert folgende Funktionen:
\begin{itemize}
	\item graphisches Erstellen / Bearbeiten eines Distanzgraphen
	\item Erstellen / Bearbeiten eines Graphen mittels Kanten- / Knotenliste oder Adjazenzmatrix
	\item Speichern und Laden von Graphen
	\item Visualisierung des Dijkstra-Algorithmus
\end{itemize}

\section{Modul starten}
Um das Modul zu starten, w�hlt man im Men� \textsc{<Datei>} das Submen� \textsc{<Neu>} und dann den Men�befehl \textsc{\dijkstra}. Im Hauptfenster erscheint nun die Oberfl�che des \dijkstra - Moduls im Eingabe-Modus.

\section{Symbolleiste}
Die Symbolleiste stellt die Funktionen \textsc{Speichern, Speichern unter, R�ckg�ngig} und \textsc{Wiederherstellen} bereit.
\newpage
\section{Programmstart - Der Willkommensbildschirm}
Nach Starten des Hauptprogramms \jalgo k�nnen Sie �ber den Button <\textsc{Neu}> oder mit dem Men�punkt \textsc{<Datei>$\rightarrow$<Neu>$\rightarrow$<\avl>} eine neue Instanz des Moduls \avl �ffnen. Anschlie�end �ffnet sich der Willkommensbildschirm des Moduls, der Ihnen verschiedene M�glichkeiten er�ffnet.\\
\bigskip
\centerpic{avl/welcomscreen}{0.5}{Der Willkommensbildschirm des Moduls \avl}
\bigskip

\subsectionicon{Baum laden}{avl/welcome_load}
Mit Klick auf das Ordner-Symbol �ffnet sich ein Dialogfenster, in dem Ihnen die M�glichkeit 
gegeben wird, eine \emph{"`*.jalgo"'} - Datei auszuw�hlen, in welcher ein Baum gespeichert wurde.\\
Im Prinzip ist die Bedeutung dieses Buttons die gleiche wie des <\textsc{�ffnen}>-Buttons in der Werkzeugleiste. Der Unterschied besteht darin, dass der Button in der Werkzeugleiste eine neue Modulinstanz �ffnet, in welcher die Datei geladen wird, der Button im Startbildschirm von \avl jedoch die Datei in die aktuell ge�ffnete Modulinstanz l�dt.

\subsectionicon{Baum von Hand erstellen}{avl/welcome_manual}
Mit Klick auf das Hand-Symbol gelangen Sie sofort zur leeren Arbeitsfl�che des Moduls \avl.
Sie k�nnen jetzt mit der knotenweisen Generierung eines neuen Suchbaumes beginnen.

\subsectionicon{Zufallsbaum erstellen lassen}{avl/welcome_random}
Mit Klick auf das W�rfel-Symbol beginnen Sie die Generierung eines zuf�llig erzeugten Suchbaumes. In dem folgenden Dialogfenster k�nnen Sie verschiedene Daten zum Baum und die Art der Visualisierung festlegen. 
\centerpic{avl/rgd}{1}{Eingabe der Zufallsbaumdaten}
\begin{itemize}
	\item {\bf Anzahl der Knoten}\\
	 Geben Sie hier die Anzahl der Knoten ein. Der entstehende Baum muss mindestens einen Knoten enthalten, h�chstens aber 99. 
	\item {\bf AVL-Eigenschaft}\\
	 Aktivieren Sie dieses K�stchen, wenn der zu erstellende Baum die AVL-Eigenschaft besitzen soll.
	\item {\bf Visualisierung}\\
	W�hlen Sie hier die Art der Visualisierung der Erstellung aus.
	\begin{itemize}
		\item {\sc keine}\\ Der Baum wird sofort erstellt.
		\item {\sc schrittweise}\\ Jeder Algorithmusschritt kann von Ihnen per Hand best�tigt werden.
		\item {\sc automatisch}\\ Lassen Sie die Erstellung des Baumes als Animation ablaufen, die Geschwindigkeit ist dabei einstellbar. 
	\end{itemize}
	Haben Sie schrittweise oder automatische Visualisierung gew�hlt, k�nnen Sie den Ablauf jederzeit abbrechen. 
    Dabei wird das gerade aktive Knoteneinf�gen abgebrochen, und der Baum steht mit entsprechend weniger Knoten zur Verf�gung.
\end{itemize}

\subsectionicon{Willkommensbildschirm anzeigen}{avl/logo}
Mit Klick auf diesen Button in der Werkzeugleiste des Modulbildschirms kann der Willkommensbildschirm sp�ter jederzeit wieder angezeigt werden. Dabei werden Sie eventuell gefragt, wie Sie mit Ihren �nderungen verfahren wollen. Sollten Sie Ihre �nderungen nicht verwerfen wollen, so wird eine neue Instanz des Moduls ge�ffnet.
\centerpic{avl/clearmessage}{1}{Dialog mit der Frage, ob der ganze Baum gel�scht werden soll.}

\bigskip
\section{Der Arbeitsbereich}
\label{sec:DerArbeitsbereich}

Der Arbeitsbereich ist untergliedert in drei Bereiche, die Ihnen Zugriff auf alle wesentlichen Funktionen in den Algorithmus-Phasen erm�glicht:

\bigskip
\centerpic{kmp/phase1screen}{0.5}{Phase 1 Bildschirm}
\bigskip

   
\begin{enumerate}
	\item  \textbf{Steuerung}\newline
      Hier wird der Algorithmus gestartet und gesteuert.
   \item \textbf{Visualisierung}\newline
      Hier werden die Verschiebetabelle bzw. der Text dargestellt.
   \item \textbf{Dokumentation}\newline
      Hier gibt es verschiedene Perspektiven, die miteinander kombiniert werden k�nnen: 			Quellcode, Text, Protokoll.
\end{enumerate}

\section{Anzeigeoptionen}
\label{sec:Anzeigeoptionen}

\subsection{Skalierung}
\label{sec:Skalierung}

Die Gr��e der Elemente des Visualisierungs- und Dokumentationsbereichs kann eingestellt werden, klicken Sie daf�r auf den Schieberegler im Visualisierungsbereich und ziehen Sie ihn nach oben bzw. unten.

\centerpic{kmp/slide}{0.5}{Auswirkung des Schiebereglers zur Skalierung}
\bigskip
\subsection{Aufteilung der Bereiche}
\label{sec:Aufteilung der Bereiche}

Die Aufteilung zwischen dem Visualisierungs- und dem Dokumentationsbereich kann mit einem Schiebebalken ver�ndert werden. Per Klick auf die Kante k�nnen Sie die Grenze verschieben.

\centerpic{kmp/move}{0.5}{Auswirkung des Schiebereglers}
\bigskip

\subsection{Beamer - Modus}
\label{sec:BeamerModus}

Der Beamer-Modus erm�glicht das schnelle Einstellen von Anzeigeoptionen, die die Pr�sentation in Vorlesungen oder �hnlichen Veranstaltungen beg�nstigen. Dieser Modus ist zu finden unter 'Knuth Morris Pratt' => 'Beamermodus'.
Die Anzeige ist f�r die Aufl�sung 1024x768 optimiert und vergr��ert die Elemente um den Faktor 1,5. Im Dokumentationsbereich wird die Perspektive 'Code' angezeigt.
Ist der Modus aktiv, so erscheint vor diesem Men�eintrag ein H�kchen. Um den Modus wieder auszuschalten, entfernen Sie einfach den Haken per Klick.

\centerpic{kmp/beamermod}{0.5}{Der Beamermodus}
\bigskip
\subsection{Zyklus - Anzeige}
\label{sec:ZyklusAnzeige}

Im Visualisierungsbereich der Phase 1 'Generierung der Verschiebetabelle' k�nnen optional die Zyklen angezeigt werden, setzen Sie dazu das H�kchen 'Zyklen' im Steuerungsbereich. Es werden maximal drei Zyklen gleichzeitig angezeigt.

\centerpic{kmp/cycle}{0.5}{Beispiel f�r Zyklen}
\bigskip
\section{Legende}
\label{sec:Legende}

\subsection{Elemente in der Phase 'Generierung der Verschiebetabelle'}
\label{sec:ElementeInDerPhaseGenerierungDerVerschiebetabelle}

\bigskip
\centerpic{kmp/phase1screen_ex1}{0.5}{Beispiel f�r Phase 1}
\bigskip
\textit{Visualisierungsbereich}\newline
\textbf{Pfeil schwarz mit wei�em 'P' }- Zeiger auf die Variable 'patpos', die Patternposition\newline
\textbf{Pfeil wei� mit schwarzem 'V' }- Zeiger auf die Variable 'VglInd', der Vergleichsindex\newline
\textbf{schwarzer Pfeil �ber der Tabelle }- die verglichenen Zeichen\newline
\textbf{gelber Hintergrund von Zellen} - Zellenkopf\newline
\textbf{schwarzer Rahmen um Zelle in der Zeile 'Index' }- aktuell kopierte Verschiebeinformation\newline
\textbf{roter Rahmen um Zellen }- negative boolesche Bedingung\newline
\textbf{lila Strich }- Zyklen, die im Pattern auftreten
\newline\newline
\textit{Schriftfarben}\newline
\textbf{blau} - aktuell geschriebene Verschiebeinformation (entspricht Zuweisungsstatement im Code)\newline
\textbf{rot} - die verglichenen Zeichen stimmen nicht �berein (entspricht booleschem Statement im Code)\newline
\textbf{gr�n} - die verglichenen Zeichen stimmen �berein (entspricht booleschem Statement im Code)

\subsection{Elemente in der Phase 'Suchen im Suchtext'}
\label{sec:ElementeInDerPhaseSuchenImSuchtext}
 \bigskip
\centerpic{kmp/phase2screen_ex1}{0.5}{Beispiel f�r diese Phase}
 \bigskip
\textit{Visualisierungsbereich}\newline
\textbf{lila Rahmen }- das Sichtfenster, welches die aktuell betrachteten Zeichen des Textes und des Patterns justiert
\newline\newline
\textit{Schriftfarben}\newline
\textbf{blau} - aktuell geschriebene Verschiebeinformation (entspricht Zuweisungsstatement im Code)

\subsection{Im Dokumentationsbereich}
\label{sec:ImDokumentationsbereich}

\bigskip
\centerpic{kmp/phase2screen_ex2}{0.5}{Beispiel f�r diese Phase}
\bigskip
\textit{Perspektive 'Code'}\newline
\textbf{roter Hintergrund }- negative boolesche Bedingung\newline
\textbf{gr�ner Hintergrund} - positive boolesche Bedingung\newline
\textbf{blauer Hintergrund }- sonstige Statements
\newline\newline
\textit{Perspektive 'Protokoll'}\newline
\textbf{blaue Schrift} - zuletzt vollzogener Schritt\newline
\textbf{schwarze Schrift} - f�r den aktuellen Stand irrelevante vorausgegangene Schritte
\newline\newline
\textit{Perspektive 'Suchtext'}\newline
\textbf{gelber Hintergrund} - der Textausschnitt, an dem das Pattern momentan anliegt

\section{Modulfunktionen}
\label{sec:Modulfunktionen}

\subsection{Eingabe von Pattern und Text}
\label{sec:EingabeVonPatternUndText}

\subsubsection{Pattern eingeben}
\label{sec:PatternEingeben}


\centerpic{kmp/pattern1}{0.5}{Pattern manuell eingeben}
\bigskip
Das Pattern kann manuell eingegeben werden, hierbei ist die maximale L�nge von 10 Zeichen zu beachten. Das Alphabet kann aus beliebigen Zeichen bestehen. Die Eingabe erfolgt im Steuerungsbereich in die Eingabezeile 'Pattern'. Um das Pattern zu �bernehmen klicken Sie auf 'Setzen'.

\subsubsection{Suchtext eingeben}


Der Text unterliegt keinen Beschr�nkungen und kann manuell eingegeben oder aus einer vorhandenen *.txt-Datei importiert werden.
Die Eingabe erfolgt im Steuerungsbereich, klicken Sie hierf�r auf 'Eingeben'.

\centerpic{kmp/text1}{0.5}{Text manuell eingeben}
\bigskip
Es �ffnet sich ein Fenster, in dem Sie den Text manuell eingeben (oder auch per Copy und Paste einf�gen) oder eine *.txt-Datei laden k�nnen. Um den Text zu �bernehmen klicken Sie auf 'Anwenden'.

\centerpic{kmp/text1}{0.5}{Text laden oder eingeben}
\bigskip
\subsection{Generieren von Pattern und passenden Texten}
\label{sec:GenerierenVonPatternUndPassendenTexten}

\subsubsection{Pattern eingeben}

\centerpic{kmp/pattern2}{0.5}{Pattern generieren lassen}
\bigskip
Neben der manuellen Eingabe des Patterns gibt es die M�glichkeit, ein Pattern generieren zu lassen. Klicken Sie daf�r auf den Knopf 'Zufall' (im Steuerungsbereich neben der Eingabezeile 'Pattern') und w�hlen Sie im folgenden Fenster die Kardinalit�t des zu nutzenden Alphabets und die gew�nschte L�nge des Patterns aus. Das Pattern wird mit Klick auf 'Anwenden' geniert und Sie kehren zum Arbeitsbereich zur�ck.

\centerpic{kmp/pattern3}{0.5}{Pattern generieren lassen}
\bigskip
\subsubsection{Suchtext eingeben}

\centerpic{kmp/text3}{0.5}{Suchtext generieren lassen}
\bigskip
Um einen zum Pattern passenden Text generieren zu lassen, klicken Sie auf 'Generieren' (im Steuerungsbereich neben 'Text'). Dadurch wird eine Text erstellt, der das Pattern enth�lt und nur aus den im Pattern genutzten Zeichen besteht. Dieser Text wird sofort �bernommen, Sie haben aber die M�glichkeit, ihn zu bearbeiten, indem Sie auf 'Eingeben' klicken.


\subsection{Generierung der Verschiebetabelle}
\label{sec:GenerierungDerVerschiebetabelle}


Wenn ein Pattern gesetzt wurde, kann die Verschiebetabelle erstellt werden. Um den Algorithmus zu starten und zu steuern, benutzen Sie die Pfeile im Steuerungsbereich.

\centerpic{kmp/control}{0.5}{Algorithmussteuerung}
\bigskip
Das Pattern kann nun noch erweitert werden ohne den aktuellen Algorithmus zu unterbrechen, geben Sie hierf�r das gew�nschte Zeichen in das Eingabefeld "Erweiterung" im Steuerungsbereich ein. Sollte das Pattern bereits die maximale L�nge haben, ist eine Erweiterung nicht m�glich.

\centerpic{kmp/expand}{0.5}{Pattern erweitern}
\bigskip
Nachdem die Verschiebetabelle erstellt wurde, k�nnen Pattern und Verschiebetabelle f�r eine Suche im Text genutzt werden. Klicken Sie hierf�r auf das Feld 'Weiter zu Phase 2', welches erst am Ende des Algorithmus im Steuerungsbereich erscheint.

\centerpic{kmp/go_p2}{0.5}{Von Phase 1 in Phase 2 wechseln}
\bigskip
\subsection{Suchen im Suchtext}
Erst wenn Pattern und Text gesetzt sind, kann das Anwenden der Verschiebetabelle an einem Text durchgef�hrt werden. Um Pattern und Text zu setzen, haben Sie verschiedene M�glichkeiten, siehe: Eingabe von Pattern und Text , Generieren von Pattern und passenden Texten.
Um den Algorithmus zu starten und zu steuern, benutzen Sie die Pfeile im Steuerungsbereich.

\centerpic{kmp/control}{0.5}{Algorithmussteuerung}
\bigskip
Sollten Sie w�hrend der Durchf�hrung des Algorithmus das Pattern und/oder den Text �ndern, wird die aktuelle Durchf�hrung des Algorithmus unterbrochen.
\bigskip
Der Suchvorgang wird beendet, wenn das Pattern gefunden wurde oder das Textende erreicht wurde, es erscheint eine Meldung dazu im Steuerungsbereich.

\centerpic{kmp/endnegativ}{0.5}{Nachricht, dass Pattern nicht gefunden wurde}
\bigskip
\centerpic{kmp/endpositiv}{0.5}{Nachricht, dass Pattern gefunden wurde}
\bigskip
\subsection{�ffnen einer KMP-Sitzung}
\label{sec:�ffnenEinerKMPSitzung}


Mit Klick auf das Ordner-Symbol �ffnet sich ein Dialogfenster, in dem Ihnen die M�glichkeit gegeben wird, eine '*.jalgo' - Datei auszuw�hlen, in welcher eine KMP-Sitzung gespeichert wurde.

\centerpic{kmp/load}{0.5}{Dialogfenster zum Laden einer Sitzung}
\bigskip

\subsection{Pr�sentation von Lernbeispielen}
Es stehen Ihnen zehn Beispiele zur Auswahl, die jeweils eine besondere Eigenschaft des KMP-Algorithmus repr�sentieren. Wenn Sie mit dem Mauszeiger �ber die Beispiele fahren, werden diese Eigenschaften kurz beschrieben. W�hlen Sie das gew�nschte Lernbeispiel aus und klicken Sie auf 'Laden' um das Beispiel zu starten.\newline
Die Beispiele bestehen aus Pattern- und passendem Text.

\centerpic{kmp/learning}{0.5}{Pr�sentation von Lernbeispielen}
\bigskip
Die Steuerung der Lernbeispiele erfolgt durch die Pfeile im Steuerungsbereich.


\section{Algorithmussteuerung}
Aufgabe des Moduls \avl ist es, Baumalgorithmen, wie das Einf�gen und L�schen von Knoten, zu visualisieren. Jeder Algorithmus ist in verschiedene Teilschritte unterteilt, die nacheinander angezeigt werden. Das Visualisieren erfolgt dabei durch das Zeichnen des Baumes, durch die Erkl�rung der Schritte im Dokumentationsbereich und im Logbuch und durch die Neuberechnung der baumspezifischen Daten, die im Infobereich pr�sentiert werden.\\
Nachdem Sie einen Algorithmus gestartet haben, verweilt er in einem Initialzustand und wartet auf Ihre Eingabe. Nun haben Sie die M�glichkeit, den Algorithmus in kleinen oder gro�en Schritten zu durchlaufen; Sie k�nnen ihn sofort beenden oder direkt abbrechen. Daf�r bietet die Algorithmussteuerung die entsprechenden Werkzeuge.

\subsection{Schritt-Pfeile}
Mittels der Schritt-Pfeile steuern Sie die Abfolge der Einzelschritte und bekommen so eine detaillierte Sicht auf die Arbeitsweise des Algorithmus. Das Programm bietet Ihnen die M�glichkeit, einen Teilschritt r�ckg�ngig zu machen und damit gewisse Abl�ufe zu wiederholen. Die Schritt-Pfeile, welche die R�ckg�ngigfunktion anbieten, weisen in ihrer Richtung nach links und sind dadurch intuitiv von den Vorw�rts-Pfeilen zu unterscheiden.\\
Zus�tzlich gibt es f�r jede Richtung einen gro�en und einen kleinen Schritt, der per Knopfdruck ausgef�hrt wird.

Kleine Schritte beim Einf�gen eines Knotens stellen Schl�sselvergleiche, Balancenberechnungen und Rotationen dar. Gro�e Schritte hingegen sind zum Beispiel das Suchen der Einf�gestelle, das Einf�gen an dieser und die gesamte Balancenaktualisierung.

\subsubsectiondoubleicon{Einzel-Schritt-Pfeile}{avl/icon_undo}{avl/icon_perform}
Ein Klick auf diese Buttons realisiert einen kleinen Algorithmusschritt zur�ck bzw. nach vorn.
\subsubsectiondoubleicon{Block-Schritt-Pfeile}{avl/icon_undo_blockstep}{avl/icon_perform_blockstep}
Ein Klick auf diese Buttons realisiert einen gro�en Schritt zur�ck bzw. nach vorn. Sollte der Algorithmusablauf an eine Stelle geraten, an der es nur noch einen kleinen Schritt nach vorn bzw. zur�ck gibt, so hat der Block-Schritt die selbe Funktionalit�t wie ein Einzel-Schritt.

\subsectiondoubleicon{Abbruch und Beenden-Buttons}{avl/icon_abort}{avl/icon_finish}
Klicken Sie auf den Beenden-Button \raisebox{-1ex}{\includegraphics[scale=0.8]{\pfad avl/icon_finish}} um den laufenden Algorithmus bis zum Ende auszuf�hren.\\
Klicken Sie auf den Abbruch-Button \raisebox{-1ex}{\includegraphics[scale=0.8]{\pfad avl/icon_abort}} um den laufenden Algorithmus abzubrechen. Der Baum hat danach den gleichen Status wie vor Beginn des Algorithmus.\\
Ist ein Algorithmus beendet, so steht Ihnen diese Option nicht mehr zur Verf�gung, weil nur der
\textit{laufende} Algorithmus abgebrochen werden kann.

\subsection{Animationsgeschwindigkeit}
Beim Generieren eines Zufallsbaumes haben Sie die Option, den Ablauf der Baumerzeugung als Animation ablaufen zu lassen. Starten Sie in diesem Modus, so beginnt die Animation sofort und kann mit dem Geschwindigkeitsregler schneller oder langsamer abgespielt werden. Zu Beginn steht dieser auf der mittleren Position. Verschieben Sie den Regler nach links, um die Animation zu verlangsamen bzw. nach rechts, um sie zu beschleunigen.\\
Eine Animation der anderen Algorithmenabl�ufe ist in dieser Version von \avl nicht integriert.

\bigskip
\centerpic{avl/animregler}{1}{Der Regler f�r die Animationsgeschwindigkeit}

\bigskip
\section{Dokumentation}
Da das Modul \avl vor allem zu Lehr- und Lernzwecken eingesetzt werden soll, ist eine detaillierte Dokumentation der Algorithmen unumg�nglich. F�r die Einzelheiten des Algorithmustextes steht ein Auszug aus dem Vorlesungsskript von Prof. Vogler zur Verf�gung. Ein Logbuch in der rechten unteren Ecke des Bildschirmes f�hrt Protokoll �ber den Stand und die Beschaffenheit des einzelnen Algorithmusteilschrittes.\\
Zu guter Letzt wird ein Infobereich angeboten, in dem wichtige Baumdaten zusammengefasst sind.

\subsection{Skript}
Der Dokumentationsbereich, der das Skript enth�lt, befindet sich am unteren Bildschirmrand. Es handelt sich hierbei um einen Auszug des Skripts zur Vorlesung "`Algorithmen und Datenstrukturen"' von Prof. Vogler (TU Dresden), Version vom 2. Oktober 2003. Im Rahmen dieser Vorlesung soll das Modul vorwiegend eingesetzt werden.\\
Bei dem jeweils aktuellen Algorithmustext handelt es sich um die Aktion, die als n�chstes im Ablauf des Algorithmus erfolgen wird. Sie wird rot markiert angezeigt.

\subsection{Logbuch}
Das Logbuch ist eine weitere M�glichkeit, den Ablauf des Algorithmus zu verfolgen. Es bezieht sich in erster Linie auf baumspezifische Daten und verwendet zum Beispiel konkrete Schl�sselwerte, anhand deren die Aktionen des Algorithmus besser verstanden werden sollen.\\
Auch hier wird der aktuelle Eintrag rot markiert dargestellt. Dieser bezieht sich aber auf die zuletzt ausgef�hrte Aktion.

\subsection{Infobereich}
Der Infobereich ist haupts�chlich daf�r gedacht, Ihnen schnell wichtige Daten des Baumes bereit zu stellen. Hier finden Sie folgende Punkte:
\begin{itemize}
	\item \textsc{Anzahl der Knoten}\\
		Dieser Punkt fasst f�r Sie die Anzahl der Knoten im Baum zusammen.
	\item \textsc{Baumh�he}\\
		Hier finden Sie die Anzahl der Level des Baumes.
	\item \textsc{Durchschnittliche Suchtiefe}\\
		Dieser Wert berechnet sich durch die Summe der Level aller Knoten geteilt durch die Anzahl dieser. Der Wert ist ein Indiz daf�r, wie gut der Baum ausbalanciert ist bzw. wie gro� der Suchaufwand im Durchschnitt ist.
\end{itemize}

\bigskip
\section{Zusatzfunktionen}
Dieses Kapitel widmet sich den Eastereggs des Moduls \avl.\\
Sollten Sie sich lieber selber gerne auf die Suche nach diesen Zusatzfunktionen machen wollen, so �berspringen Sie besser dieses Kapitel.\\
F�r alle Anderen folgt nun eine �bersicht zum Baumnavigator und dem Beamermodus.

\subsection{Navigator}
Der Navigator ist eine kleine, versteckte Zusatzfunktion, die die Arbeit mit gro�en B�umen erheblich vereinfachen kann. Er stellt eine willkommene Hilfe f�r das Scrollen der Zeichenfl�che dar, ist aber nicht so einfach zu finden.
\begin{itemize}
	\item Wenn der Baum, der auf der Zeichenfl�che angezeigt wird, zu gro� f�r diese wird, so erscheinen Schiebebalken, mit denen Sie den Bildausschnitt verschieben k�nnen.
	\item Klicken Sie nun auf das kleine Quadrat in der rechten unteren Ecke der Zeichenfl�che, genau zwischen den beiden Schiebebalken. Halten Sie dabei die linke Maustaste gedr�ckt.
	\item Eine kleine �bersichtskarte des Baumes mit einem Ausschnittfenster erscheint. Bewegen Sie die Maus (mit gedr�ckter Taste) und das Ausschnittfenster, das den Bildschirminhalt der Zeichenfl�che repr�sentiert, folgt Ihren Bewegungen.
\end{itemize}
\bigskip
\begin{center}
	\includegraphics[scale=0.7]{\pfad avl/navigator1} \hfill
	\includegraphics[scale=0.7]{\pfad avl/navigator2} \\
	Ein Klick auf das kleine K�stchen... \hfill ...�ffnet den Navigator!
\end{center}

\subsectionicon{Beamermodus}{avl/icon_beamer}
Der Beamermodus ist in erster Linie f�r die Pr�sentation in Vorlesungen oder �hnlichen Veranstaltungen gedacht. Ist dieser Modus aktiv, so werden die Knoten des Baumes und die Eintr�ge des Logbuches vergr��ert dargestellt. Der Algorithmustext aus dem Skript von Prof. Vogler bleibt dabei unver�ndert, weil davon ausgegangen wird, dass die interessierten Studenten der Vorlesung �ber ein (eventuell aktuelleres) Skript verf�gen.\\
Sie erreichen den Beamermodus �ber den Men�punkt \textsc{<\avl>} $\rightarrow$ \textsc{<Beamermodus>}. Ist der Modus aktiv, so erscheint neben diesem Men�eintrag ein H�kchen. Um den Modus wieder auszuschalten, entfernen Sie einfach den Haken per Klick.
\bigskip
\centerpic{avl/beamermenu}{0.5}{Das Men� \textsc{<\avl>} mit dem Eintrag \textsc{<Beamermodus>}}

\newpage
\section{Impressum}
Die \jalgo Software wurde im Sommersemester 2004 von der Praktikumsgruppe SWT04-PROG1 im Rahmen des externen Softwarepraktikums entwickelt. Mitwirkende waren die
\paragraph{Teammitglieder}
\begin{itemize}
	\item Michael Pradel --- Chief of Algorithms
	\item Cornelius Hald --- Chief of Framework
	\item Malte Blumberg
	\item Stephan Creutz
	\item Christopher Friedrich
	\item Anne Kersten
	\item Hauke Menges
	\item Babett Schalitz
	\item Benjamin Scholz
	\item Marco Zimmerling
\end{itemize}
Die Webseite des Praktikums finden Sie unter \url{http://web.inf.tu-dresden.de/~swt04-p1/}.\\
Komplette �berarbeitung erfuhr die Software und die Dokumentation unter anderem durch Alexander Claus und Matthias Schmidt. Weitergehende Informationen �ber \jalgo erhalten Sie unter \url{http://j-algo.binaervarianz.de/}.

\part{Der Anhang}
\begin{appendix}
\newtheorem{alg}{Algorithmus}

\chapter{\label{appendix_avl_A}Einleitung zu Datenstrukturen}
\begin{tabbing}
\textbf{Anmerkung:} \=Der Autor der folgenden Seiten (Kapitel \ref{appendix_avl_A}, \ref{appendix_avl_B} und \ref{appendix_avl_C} dieses Anhangs) ist\\
\> Jean Christoph Jung (Teammitglied AVL-Modul)
\end{tabbing}

Eine h�ufige Anwendung auf gro�en Datenmengen ist das Suchen: Das Wiederfinden eines bestimmten Elements oder bestimmter Informationen aus einer gro�en Menge fr�her abgelegter Daten. Um die Suche zu vereinfachen, werden den (m�glicherweise sehr komplexen) gro�en Datens�tzen eineindeutige Suchschl�ssel (Keys) zugeordnet. Der Vergleich zweier solcher Schl�ssel ist in der Regel viel schneller als der Vergleich zweier Datens�tze. Wegen der Schl�sselzuordnung reicht es aus, alle vorkommenden Algorithmen nur auf den Schl�sseln zu betrachten; in Wirklichkeit verweisen erst die Schl�ssel auf die Datens�tze.\\
Eine grundlegende Idee ist nun, die Daten in Form einer Liste oder eines Feldes abzulegen. Diese Datenstruktur ist h�chst einfach. Jedoch ist der Aufwand f�r das Suchen relativ hoch, n�mlich $O(n)$. Genauer gesagt ben�tigt man f�r eine erfolglose Suche $n$ Vergleiche (bei $n$ Elementen in der Datenstruktur), denn man muss jedes Element �berpr�fen, und f�r eine erfolgreiche Suche durchschnittlich $(n+1)/2$ Vergleiche durchf�hren, da nach jedem Element mit der gleichen Wahrscheinlichkeit gesucht wird.\\
Eine Verbesserung dieses Verfahrens w�re, die Daten sortiert abzulegen, was zu einer bin�ren Suche f�hrt. Die Suche hat jetzt nur noch Komplexit�t $O(\log_2{n})$, da bei jedem Suchschritt das Feld halbiert wird. Der Nachteil ist, dass durch die Sortierung das Einf�gen erschwert wird, da unter Umst�nden viele Datens�tze bewegt werden m�ssen. Das Verfahren sollte also nur angewandt werden, wenn sehr wenige oder gar keine Einf�geoperationen ausgef�hrt werden m�ssen. Dann k�nnen die Daten anfangs mit einem schnellen Verfahren sortiert werden und m�ssen danach nicht mehr ver�ndert werden.\\
Eine weitere Datenstruktur, die der Suchb�ume, wollen wir hier vorstellen.

\chapter{\label{appendix_avl_B}Suchb�ume}
Suchb�ume sind bin�re B�ume (jeder Knoten hat h�chstens 2 Kinder) mit der Eigenschaft:\\
F�r jeden Knoten gilt: alle Schl�ssel im rechten Teilbaum sind gr��er als der eigene Schl�ssel und alle Schl�ssel im linken Teilbaum sind kleiner. Diese Eigenschaft wird hier immer Suchbaumeigenschaft genannt.

\centerpic{avl/bsp_suchbaum}{4.5}{Ein Beispiel f�r einen Suchbaum} 

An dieser Stelle noch eine Bemerkung zu den Schl�sseln: Die Schl�ssel k�nnen Elemente einer beliebigen Menge sein, unter der Bedingung, dass auf dieser Menge eine Ordnungsrelation definiert ist. Die Ordnungsrelation wird offensichtlich f�r die Suchbaumeigenschaft ben�tigt, da dort die Begriffe "`kleiner"' und "`gr��er"' vorkommen. Eine h�ufig verwendete Menge sind die nat�rlichen Zahlen mit ihrer normalen Ordnungsrelation $\leq$. Alle Operationen auf Suchb�umen kann man sich also anhand der nat�rlichen Zahlen vorstellen.\\
Aus der Suchbaumeigenschaft kann man sich leicht rekursive Algorithmen f�r das Einf�gen und Suchen in einem Suchbaum herleiten:

\begin{alg} \label{search}
	(Suchen eines Schl�ssels $s$ in einem Suchbaum)
	\begin{enumerate}
		\item Falls Teilbaum leer, dann Schl�ssel nicht im Baum vorhanden
		\item Falls $s$ gleich dem Schl�ssel des aktuellen Knoten, dann Suche erfolgreich.
		\item Falls $s$ gr��er als Schl�ssel des aktuellen Knotens, dann suche (rekursiv) $s$ im rechten Teilbaum.
		\item Falls $s$ kleiner als Schl�ssel des aktuellen Knotens, dann suche (rekursiv) $s$ im linken Teilbaum.
	\end {enumerate}
\end{alg}

\newpage
\begin{alg} \label{insert}
	(Einf�gen eines Schl�ssels $s$ in einen Suchbaum)
	\begin {enumerate}
		\item Falls Teilbaum leer, dann neuen Schl�ssel hier einf�gen.
		\item Falls $s$ gleich dem Schl�ssel des aktuellen Knotens, dann Schl�ssel bereits vorhanden, Einf�gen nicht n�tig.
		\item Falls $s$ gr��er als Schl�ssel des aktuellen Knotens, dann f�ge (rekursiv) $s$ in den rechten Teilbaum ein.
		\item Falls $s$ kleiner als Schl�ssel des aktuellen Knotens, dann f�ge (rekursiv) $s$ in den linken Teilbaum ein.
	\end {enumerate}
\end{alg}

Mit Algorithmus \ref{insert} ergibt sich folgende Eigenschaft der Struktur von Suchb�umen: im Gegensatz zur sortierten Liste h�ngt die Struktur eines Baumes davon ab, in welcher Reihenfolge die Elemente eingef�gt werden. So erh�lt man verschiedene B�ume, wenn man $1, 2, 3, 4$ in dieser Reihenfolge und in der Reihenfolge $3, 2, 4, 1$ einf�gt.

\centerpic{avl/12345}{4}{Unterschiedliche Suchb�ume bei unterschiedlicher Einf�gereihenfolge} 

Im ersten Fall erh�lt man einen Suchbaum, der zur linearen Liste entartet ist. Das ist nicht nur in dieser speziellen Reihenfolge so, es gibt viele M�glichkeiten einen entarteten Baum zu erzeugen. Der Algorithmus hat also zwei Nachteile: zum einen kann der Suchaufwand linear zur Anzahl der Knoten im Baum sein, was keine Verbesserung zur linearen Liste darstellt; zum anderen h�ngt die G�te des Verfahrens von der Eingabefolge ab. Der Vorteil von Suchb�umen wird klar, wenn man einen "`vollen"' Baum betrachtet, d.h. alle Pfade zu Bl�ttern haben dieselbe L�nge. Dann hat sowohl das Suchen (unabh�ngig davon, ob der Schl�ssel im Baum vorhanden ist) als auch das Einf�gen eine Komplexit�t von $O(\log{n})$.

\centerpic{avl/vollerbaum}{5}{Ein Beispiel f�r einen vollen Baum}
\medskip
Es gibt einige Algorithmen, bei denen der Einf�gealgorithmus so modifiziert ist, dass die entarteten F�lle vermieden werden und immer nahezu volle B�ume entstehen. Einen davon, den Algorithmus nach Adelson-Velskij und Landis (AVL), werden wir sp�ter betrachten.

Doch zun�chst wollen wir noch eine weitere wichtige Operation auf Suchb�umen untersuchen: das L�schen. Leider ist es nicht ganz so einfach wie Suche und Einf�gen.\\
Zuerst muss der zu l�schende Schl�ssel gesucht werden. Ist der betreffende Knoten ein Blatt, kann er einfach entfernt werden. Hat der zu l�schende Knoten nur ein Kind, kann er durch dieses ersetzt werden. Der schwierige Fall ist, wenn er zwei Kinder hat. Damit die Suchbaum\-eigenschaft erhalten bleibt, muss man ihn durch den n�chst gr��eren Schl�ssel ersetzen. Der n�chst gr��ere Schl�ssel befindet sich offensichtlich im rechten Teilbaum.

\centerpic{avl/remove}{5}{L�schen eines Knoten im Suchbaum}
\medskip
Nach dem Ersetzen bleibt die Suchbaumeigenschaft erhalten, weil alle Schl�ssel aus dem linken Teilbaum ohnehin kleiner sind als die aus dem rechten. Au�erdem sind auch alle Schl�ssel aus dem rechten Teilbaum gr��er, sonst w�re es nicht der n�chst gr��ere Schl�ssel gewesen. Aus diesen �berlegungen erh�lt man folgende verbale Beschreibung des L�schen-Algorithmus:

\newpage
\begin{alg} \label{delete}
	(L�schen eines Schl�ssels $s$ aus einem Suchbaum)
	\begin{enumerate}
		\item Suche s nach Algorithmus \ref{search}. Falls s nicht im Baum enthalten, terminiert der Algorithmus.
		\item \begin{enumerate}
			\item Ist der zu l�schende Knoten ein Blatt, dann entferne ihn einfach aus dem Baum.
			\item Hat der zu l�schende Knoten nur ein Kind, dann ersetze ihn durch dieses.
			\item Sonst suche den kleinsten Schl�ssel im rechten Teilbaum: Gehe zum rechten Kind und dann immer zum linken Teilbaum, solange dieser nicht leer ist. Ersetze den zu l�schenden Schl�ssel durch den des so gefundenen Knotens. Ersetze den gefundenen Knoten durch sein rechtes Kind.
			\end{enumerate}
	\end{enumerate}
\end{alg}

Man kann anstelle des n�chst gr��eren Schl�ssels genausogut den n�chstkleineren nehmen, der Algorithmus funktioniert trotzdem. Zur Komplexit�t ist zu sagen, dass der Algorithmus maximal $h$ Vergleiche macht, wobei $h$ die H�he des Baumes ist. Auch hier ist also die Komplexit�t abh�ngig von der Struktur des Baumes.

\chapter{\label{appendix_avl_C}AVL-B�ume}
AVL-B�ume (benannt nach Adelson-Velskij und Landis) sind spezielle Suchb�ume: In jedem Knoten unterscheiden sich die H�hen des linken Teilbaums und des rechten Teilbaums um h�chstens 1. Um diese Eigenschaft (AVL-Eigenschaft) abzusichern, wird f�r jeden Knoten ein Balancefaktor eingef�hrt. Der Balancefaktor ist die Differenz der H�hen des rechten Teilbaums und des linken Teilbaums. Also gilt f�r jeden AVL-Baum, dass alle Balancefaktoren aus $\{-1,0,1\}$ sind. B�ume mit AVL-Eigenschaft sind niemals als lineare Liste entartet (sofern sie denn mehr als 2 Knoten haben), sondern sind immer fast vollst�ndig. Zwischen der H�he $h$ eines Baumes und der Anzahl $n$ seiner Knoten besteht folgender Zusammenhang: $h\leq 2 \cdot \log_2{n}$. Das bedeutet, dass f�r das Suchen logarithmische Komplexit�t garantiert werden kann (das Suchen erfolgt gem�� Algorithmus \ref{search}).

\centerpic{avl/avlbsp}{5}{Ein Beispiel f�r einen AVL-Baum mit Balancen}
\medskip
Jetzt muss noch untersucht werden, wie gro� der zus�tzliche Aufwand beim Einf�gen ist, um die AVL-Eigenschaft zu wahren. In jedem Fall wird der neue Knoten als Blatt eingef�gt (nach demselben Algorithmus wie bei Suchb�umen). Dabei kann sich der Balancefaktor �ndern. Allerdings kann man sich leicht klarmachen, dass das nur entlang des Suchpfades passieren kann. Die Aktualisierung von Balancefaktoren erfolgt von der Einf�gestelle zur Wurzel. Hier der Algorithmus:

\newpage
\begin{alg} \label{avlinsert}
	(Algorithmus zum Einf�gen eines Elements x in einen AVL-Baum)
	\begin{enumerate}
		\item F�ge das neue Element x als direkten Nachfolger des Knotens n als Blatt ein, sodass die Suchbaumeigenschaft erf�llt bleibt. Aktualisiere n.balance.
		\item Setze n auf den Vorg�ngerknoten von n.
			\begin{enumerate}
				\item Falls x im linken Unterbaum von n eingef�gt wurde
					\begin{enumerate}
						\item wenn n.balance==1 dann n.balance=0 und gehe nach 3.
						\item wenn n.balance==0, dann n.balance=-1 und gehe nach 2.
						\item wenn n.balance==-1 und 
							\begin{itemize}
								\item wenn n.left.balance==-1, dann Rechts(n)-Rotation.
								\item wenn n.left.balance==1 dann Links(n.left)-Rechts(n)-Rotation.
							\end{itemize}
							Gehe zu 3.
					\end{enumerate}
				\item Falls x im rechten Unterbaum von n eingef�gt wurde
					\begin{enumerate}
						\item wenn n.balance==-1 dann n.balance=0 und gehe nach 3.
						\item wenn n.balance==0, dann n.balance=1 und gehe nach 2.
						\item wenn n.balance==1 und
							\begin{itemize}
								\item wenn n.left.balance==1, dann Links(n)-Rotation.
								\item wenn n.left.balance==-1 dann Rechts(n.left)-Links(n)-Rotation.
							\end{itemize}
							Gehe zu 3.
					\end{enumerate}
			\end{enumerate}
		\item Gehe zur�ck zur Wurzel.
	\end{enumerate}
\end{alg}

Zur Analyse dieses Algorithmus: Das reine Einf�gen erfolgt gem�� Einf�gen im Suchbaum (Algorithmus \ref{insert}), allerdings ist hier sichergestellt, dass sich die H�he logarithmisch zur Anzahl der Knoten verh�lt, d.h. auch der Aufwand f�r das Einf�gen ist garantiert logarithmisch. Wie gesagt k�nnen sich jedoch Balancefaktoren ge�ndert haben, sodass die AVL-Eigenschaft nicht mehr erf�llt ist. Das wird durch sogenannte Rotationen behoben. Es gibt zwei Typen von Rotationen -- Linksrotation und Rechtsrotation, jeweils um einen Knoten n.

\centerpic{avl/rotateleft}{5}{Linksrotation um den Knoten 65}  \medskip
\centerpic{avl/rotateright}{5}{Rechtsrotation um den Knoten 58}  \medskip

Falls durch das Einf�gen irgendwo ein Balancefaktor $2$($-2$) entsteht (gr��ere �nderungen k�nnen beim Einf�gen eines Knotens offensichtlich nicht auftreten), hei�t das, dass sich im rechten (linken) Teilbaum die H�he um 1 erh�ht hat. Durch Rotation(en) wie im Algorithmus angegeben, wird aber genau diese H�he wieder reduziert. Somit ist klar, dass man maximal zweimal rotieren muss, der Aufwand ist also noch ertr�glich, im Gegensatz zum L�schen, wie man gleich sehen wird.

Genauso wie Einf�gen ver�ndert auch L�schen eines Knotens aus einem AVL-Baum die Balancefaktoren, also m�ssen auch hier Rotationen ausgef�hrt werden. Hier der Algorithmus:

\begin{alg} \label{avldelete}
	L�schen eines Knotens aus einem AVL-Baum)
	\begin{enumerate}
		\item L�sche den Knoten analog zum L�schen im Suchbaum (Algorithmus \ref{delete}). Falls der Knoten ein Blatt war oder nur einen linken Nachbarn hatte, setze aktuellen Knoten auf den Vater. Sonst setze aktuellen Knoten auf den Vater des Knotens mit dem n�chst gr��eren Schl�ssel.
		\item Berechne den Balancefaktor des aktuellen Knotens neu. Falls
			\begin{enumerate}
				\item Balance 2 und rechte Balance -1, dann Rechts(n.right)-Links(n)-Rotation.
				\item Balance 2 und rechte Balance nicht -1, dann Links(n)-Rotation.
				\item Balance -2 und linke Balance 1, dann Links(n.left)-Rechts(n)-Rotation.
				\item Balance -2 und linke Balance nicht 1, dann Rechts(n)-Rotation.
				\item sonst keine Rotation.
			\end{enumerate}
			Wiederhole diesen Schritt solange, bis die Wurzel erreicht ist.
	\end{enumerate}
\end{alg}

Auch hier wollen wir den Aufwand des Algorithmus etwas genauer untersuchen. Schritt 1 entspricht dem L�schen aus dem Suchbaum, nur garantiert mit logarithmischen Aufwand. Die Frage ist nun, ob, wie beim Einf�gen, der Algorithmus mit maximal zwei Rotationen auskommt. Leider ist das nicht der Fall. Das liegt an der erw�hnten Eigenschaft der Rotationen, sie verringern die H�he eines Teilbaums. Beim Einf�gen war das gut, da durch das Anh�ngen eines Knotens gerade die H�he vergr��ert wurde. Hier jedoch ist das unvorteilhaft, es kann passieren, dass man mehrere Rotationen auf dem Weg zur Wurzel durchf�hren muss. Die Anzahl der Rotationen ist nur durch die H�he des Baums beschr�nkt. Das ist in Anwendungsf�llen nicht w�nschenswert.

Bei zeitkritischen Anwendungen muss man also entweder auf einen anderen Algorithmus ausweichen, oder Varianten wie etwa Lazy-Delete implementieren, d.h. der Knoten wird nur als gel�scht markiert und wird sp�ter (wenn Zeit ist) aus dem Baum entfernt.

\begin{thebibliography}{99}
    \bibitem {sedgewick} R. Sedgewick, "`Algorithmen in C++"', Addison-Wesley, 5. Auflage, 1999
    \bibitem {vogler} Prof. Vogler, "`Vorlesungsskript Algorithmen, Datenstrukturen und Programmierung"', TU Dresden, 2003
    \bibitem {wiki} http://de.wikipedia.org/wiki/AVL-Baum
    \bibitem {uni-leipzig} http://dbs.uni-leipzig.de/de/skripte/ADS1/HTML/kap6-11.html
\end{thebibliography}
\end{appendix}

\end{document}