\section{Dokumentation}
Da das Modul \avl vor allem zu Lehr- und Lernzwecken eingesetzt werden soll, ist eine detaillierte Dokumentation der Algorithmen unumg�nglich. F�r die Einzelheiten des Algorithmustextes steht ein Auszug aus dem Vorlesungsskript von Prof. Vogler zur Verf�gung. Ein Logbuch in der rechten unteren Ecke des Bildschirmes f�hrt Protokoll �ber den Stand und die Beschaffenheit des einzelnen Algorithmusteilschrittes.\\
Zu guter Letzt wird ein Infobereich angeboten, in dem wichtige Baumdaten zusammengefasst sind.

\subsection{Skript}
Der Dokumentationsbereich, der das Skript enth�lt, befindet sich am unteren Bildschirmrand. Es handelt sich hierbei um einen Auszug des Skripts zur Vorlesung "`Algorithmen und Datenstrukturen"' von Prof. Vogler (TU Dresden), Version vom 2. Oktober 2003. Im Rahmen dieser Vorlesung soll das Modul vorwiegend eingesetzt werden.\\
Bei dem jeweils aktuellen Algorithmustext handelt es sich um die Aktion, die als n�chstes im Ablauf des Algorithmus erfolgen wird. Sie wird rot markiert angezeigt.

\subsection{Logbuch}
Das Logbuch ist eine weitere M�glichkeit, den Ablauf des Algorithmus zu verfolgen. Es bezieht sich in erster Linie auf baumspezifische Daten und verwendet zum Beispiel konkrete Schl�sselwerte, anhand deren die Aktionen des Algorithmus besser verstanden werden sollen.\\
Auch hier wird der aktuelle Eintrag rot markiert dargestellt. Dieser bezieht sich aber auf die zuletzt ausgef�hrte Aktion.

\subsection{Infobereich}
Der Infobereich ist haupts�chlich daf�r gedacht, Ihnen schnell wichtige Daten des Baumes bereit zu stellen. Hier finden Sie folgende Punkte:
\begin{itemize}
	\item \textsc{Anzahl der Knoten}\\
		Dieser Punkt fasst f�r Sie die Anzahl der Knoten im Baum zusammen.
	\item \textsc{Baumh�he}\\
		Hier finden Sie die Anzahl der Level des Baumes.
	\item \textsc{Durchschnittliche Suchtiefe}\\
		Dieser Wert berechnet sich durch die Summe der Level aller Knoten geteilt durch die Anzahl dieser. Der Wert ist ein Indiz daf�r, wie gut der Baum ausbalanciert ist bzw. wie gro� der Suchaufwand im Durchschnitt ist.
\end{itemize}

\bigskip