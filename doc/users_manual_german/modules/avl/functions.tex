\section{Modulfunktionen}
Alle Funktionen des Moduls \avl lassen sich �ber den Kontroll-Bereich der Arbeitsfl�che bedienen. Sie stellen die verschiedenen Baumalgorithmen dar, deren Visualisierung Aufgabe dieses Moduls ist. Grundlegend l�uft die Arbeit mit den Algorithmen immer nach dem gleichen Schema ab:
\begin{enumerate}
	\item Schl�sseleingabe
	\item Starten des Algorithmus per Klick auf den entsprechenden Button
\end{enumerate}
Es gibt nat�rlich auch Algorithmen, wie der AVL-Test, die keinen Schl�ssel ben�tigen und ohne Schritt 1 auskommen.\\
Es folgen nun die einzelnen Funktionen im Detail. 

\subsection{Schl�sseleingabe}
F�r die Eingabe der Schl�sselwerte steht ein Textfeld und ein Button f�r zuf�llige Werte zur Verf�gung. Es sind nur ganzzahlige Schl�sselwerte von 1 bis 99 erlaubt.\\
�ber dem Textfeld befindet sich eine Nachrichtenzeile, in welcher Sie auf eventuelle Fehleingaben aufmerksam gemacht werden. Hier werden sp�ter ebenfalls kurze Ergebnismeldungen zu den Algorithmen eingeblendet.

\subsection{Algorithmusfunktionen}
\subsubsection*{Knoten einf�gen}
Der eingegebene Wert wird als Schl�ssel f�r einen neuen Knoten verwendet, der in den Baum eingef�gt werden soll. Ist bereits ein Knoten mit dem gleichen Schl�ssel im Baum enthalten, so bricht der Algorithmus erfolglos ab.
\subsubsection*{Knoten suchen}
Nach dem Starten dieses Algorithmus beginnt die Suche nach dem eingegebenen Schl�ssel im Baum.
\subsubsection*{Knoten l�schen}
Nach dem eingegebenen Schl�ssel wird gesucht, und wenn ein entsprechender Knoten gefunden wurde, wird dieser aus der Baumstruktur entfernt.

\subsection{AVL-Modus}
Ist dieses K�stchen aktiviert, werden die entsprechenden Algorithmen so ausgef�hrt, dass die AVL-Eigenschaft gewahrt bleibt.\\
Achtung: Es ist keine Funktion implementiert, die an einem beliebigen Suchbaum die AVL-Eigenschaft herstellt!\\
Ist das K�stchen deaktiviert, ist es daher nicht immer ohne weiteres wieder zu aktivieren. Dazu muss zuerst getestet werden, ob der Baum die AVL-Eigenschaft hat. Es ist jedoch jederzeit m�glich, das K�stchen zu deaktivieren und einen unbalancierten Baum zu erzeugen.

\subsection{Baum auf AVL-Eigenschaft testen}
\medskip
\centerpic{avl/avltest}{1}{Hinweisfenster des AVL-Tests}
Wenn der AVL-Modus einmal deaktiviert sein sollte, so erm�glicht das Programm einen Test des Baumes auf die AVL-Eigenschaft. Dabei erfolgt eine Berechnung und Anzeige der Balancen aller Knoten und das eventuelle Markieren von Knoten, deren Balance sich nicht mehr im Rahmen der AVL-Eigenschaft bewegt.\\
Es wird ein Hinweis-Dialog ge�ffnet, der Ihnen das Ergebnis des Tests pr�sentiert. Sollte der Baum tats�chlich die AVL-Eigenschaft besitzen, so wird Ihnen angeboten, direkt in den AVL-Modus zu wechseln.

\newpage
\subsection{Baum l�schen}
Mit einem Klick auf den Button \raisebox{-1.5ex}{\includegraphics[scale=1]{\pfad avl/icon_clear}} in der Werkzeugleiste k�nnen Sie nach einer Sicherheitsabfrage die gesamte Baumstruktur l�schen und mit einer leeren Arbeitsfl�che neu beginnen.
\medskip
\centerpic{avl/cleartreemessage}{1}{Sicherheitsabfrage beim L�schen des Baumes}

\vfill