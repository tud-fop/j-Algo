\chapter{Das Modul Dijkstra}
\section{Einleitung}
Das Modul \dijkstra visualisiert den bekannten Algorithmus von E. W. Dijkstra zum Finden der k�rzesten Wege von einem Startknoten in einem Distanzgraphen. Der Algorithmus selbst ist unter anderem im Vorlesungsskript von Prof. Vogler "`Algorithmen, Datenstrukturen und Programmierung"' zu finden. Aber auch im Internet existieren zahlreiche Quellen dazu.

Soweit es m�glich gewesen ist, wurde beim Design des Moduls darauf geachtet, es weitgehend intuitiv und selbst-dokumentierend zu gestalten. Nichtsdestotrotz findet sich hier eine kurze
Einf�hrung in das \dijkstra - Modul.

\section{Funktions�bersicht}
Das Modul \dijkstra realisiert folgende Funktionen:
\begin{itemize}
	\item graphisches Erstellen / Bearbeiten eines Distanzgraphen
	\item Erstellen / Bearbeiten eines Graphen mittels Kanten- / Knotenliste oder Adjazenzmatrix
	\item Speichern und Laden von Graphen
	\item Visualisierung des Dijkstra-Algorithmus
\end{itemize}

\section{Modul starten}
Um das Modul zu starten, w�hlt man im Men� \textsc{<Datei>} das Submen� \textsc{<Neu>} und dann den Men�befehl \textsc{\dijkstra}. Im Hauptfenster erscheint nun die Oberfl�che des \dijkstra - Moduls im Eingabe-Modus.

\section{Symbolleiste}
Die Symbolleiste stellt die Funktionen \textsc{Speichern, Speichern unter, R�ckg�ngig} und \textsc{Wiederherstellen} bereit.
\newpage